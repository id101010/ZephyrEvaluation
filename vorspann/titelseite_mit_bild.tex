%
% Project documentation template
% ===========================================================================


\begin{titlepage}


% BFH-Logo absolute placed at (28,12) on A4 and picture (16:9 or 15cm x 8.5cm)
% Actually not a realy satisfactory solution but working.
%---------------------------------------------------------------------------
\setlength{\unitlength}{1mm}
\begin{textblock}{20}[0,0](28,12)
	\includegraphics[scale=1.0]{bilder/BFH_Logo_B.png}
\end{textblock}

\begin{textblock}{154}(28,48)
	\begin{picture}(150,2)
		\put(0,0){\color{bfhgrey}\rule{150mm}{2mm}}
	\end{picture}
\end{textblock}

\begin{textblock}{154}[0,0](28,50)
	\includegraphics[scale=1.0]{bilder/Zephyr-Project.jpg}			% Titelbild definieren
\end{textblock}

\begin{textblock}{154}(28,135)
	\begin{picture}(150,2)
		\put(0,0){\color{bfhgrey}\rule{150mm}{2mm}}
	\end{picture}
\end{textblock}
\color{black}

% Institution / Titel / Untertitel / Autoren / Experten:
%---------------------------------------------------------------------------
\begin{flushleft}

\vspace*{115mm}

\fontsize{26pt}{28pt}\selectfont 
\titel 				\\							% Titel aus der Datei vorspann/titel.tex lesen
\vspace{2mm}

\fontsize{16pt}{20pt}\selectfont\vspace{0.3em}
Echtzeit-OS f�r das Internet der Dinge 			\\							% Untertitel eingeben
\vspace{5mm}

\fontsize{10pt}{12pt}\selectfont
\textbf{Projektarbeit} \\									% eingeben
\vspace{3mm}

% Abstract (eingeben):
%---------------------------------------------------------------------------
\begin{textblock}{150}(28,190)
\fontsize{10pt}{12pt}\selectfont
Die Linux Foundation hat mit dem Projekt Zephyr mit der Entwicklung eines Echtzeit-Betriebssystems f�r das Internet der Dinge (IoT) begonnen.
Zephyr ist ein Open-Source-Betriebssystem mit dem Ziel ein solides OS f�r IoT Ger�te mit geringen Ressourcen bereitzustellen. Es nutzt eine echtzeitf�hige Kombination aus Nano- und Microkernel. 
Im Gegensatz zu einem Linux Kernel ben�tigt Zephyr nur zwischen 8 und 512 KByte an Arbeitsspeicher.  Aktuell werden folgenden Plattformen unterst�tzt: x86, ARM und ARC EM4
\end{textblock}

\begin{textblock}{150}(28,225)
\fontsize{10pt}{17pt}\selectfont
\begin{tabbing}
xxxxxxxxxxxxxxx\=xxxxxxxxxxxxxxxxxxxxxxxxxxxxxxxxxxxxxxxxxxxxxxx \kill
Studiengang:	\> Elektro- und Kommunikationstechnik			\\
Autoren:		\> Aaron Schmocker, David Wyss					\\
Betreuer:		\> Martin Aebersold								\\
Auftraggeber:	\> Martin Aebersold								\\
Experten:		\> Martin Aebersold								\\
Datum:			\> \versiondate									\\
\end{tabbing}

\end{textblock}
\end{flushleft}

\begin{textblock}{150}(28,280)
\noindent 
\color{bfhgrey}\fontsize{9pt}{10pt}\selectfont
Berner Fachhochschule | Haute �cole sp�cialis�e bernoise | Bern University of Applied Sciences
\color{black}\selectfont
\end{textblock}


\end{titlepage}

%
% ===========================================================================
% EOF
%
