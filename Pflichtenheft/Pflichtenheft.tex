%---------------------------------------------------------------------------
\documentclass[
	a4paper,				% paper format
	10pt,					% fontsize
	oneside,				% double-sided
	openright,				% begin new chapter on right side
	notitlepage,			% use no standard title page
	parskip=half,			% set paragraph skip to half of a line
]{scrreprt}					% KOMA-script report
%---------------------------------------------------------------------------

\raggedbottom
\KOMAoptions{cleardoublepage=plain}			% Add header and footer on blank pages


% Load Standard Packages:
%---------------------------------------------------------------------------
\usepackage[standard-baselineskips]{cmbright}
\usepackage[ngerman]{babel}								% german hyphenation
\usepackage[latin1]{inputenc}  							% Unix/Linux - load extended character set (ISO 8859-1)
%\usepackage[ansinew]{inputenc}  						% Windows - load extended character set (ISO 8859-1)
\usepackage[T1]{fontenc}								% hyphenation of words with �,� and �
\usepackage{textcomp}									% additional symbols
\usepackage{ae}											% better resolution of Type1-Fonts 
\usepackage{fancyhdr}									% simple manipulation of header and footer 
\usepackage{etoolbox}									% color manipulation of header and footer
\usepackage{graphicx}                      				% integration of images
\usepackage{float}										% floating objects
\usepackage{caption}									% for captions of figures and tables
\usepackage{booktabs}									% package for nicer tables
\usepackage{tocvsec2}									% provides means of controlling the sectional numbering
\usepackage{subfigure}

\usepackage[]{acronym}
\usepackage[acronym,toc]{glossaries}



\usepackage[style=ieee,backend=biber]{biblatex}			%Bibliography
%\addbibresource{bibliography.bib}
%---------------------------------------------------------------------------

% Load Math Packages
%---------------------------------------------------------------------------
\usepackage{amsmath}                    	   			% various features to facilitate writing math formulas
\usepackage{amsthm}                       	 			% enhanced version of latex's newtheorem
\usepackage{amsfonts}                      				% set of miscellaneous TeX fonts that augment the standard CM
\usepackage{amssymb}									% mathematical special characters
\usepackage{exscale}									% mathematical size corresponds to textsize
%---------------------------------------------------------------------------

% Package to facilitate placement of boxes at absolute positions
%---------------------------------------------------------------------------
\usepackage[absolute]{textpos}
\setlength{\TPHorizModule}{1mm}
\setlength{\TPVertModule}{1mm}
%---------------------------------------------------------------------------					
			
% Definition of Colors
%---------------------------------------------------------------------------
\RequirePackage{color}                          		% Color (not xcolor!)
\definecolor{linkblue}{rgb}{0,0,0.8}            		% Standard
\definecolor{darkblue}{rgb}{0,0.08,0.45}        		% Dark blue
\definecolor{bfhgrey}{rgb}{0.41,0.49,0.57}      		% BFH grey
%\definecolor{linkcolor}{rgb}{0,0,0.8}     				% Blue for the web- and cd-version!
\definecolor{linkcolor}{rgb}{0,0,0}        				% Black for the print-version!
%---------------------------------------------------------------------------

% Hyperref Package (Create links in a pdf)
%---------------------------------------------------------------------------
\usepackage[
	pdftex,ngerman,bookmarks,plainpages=false,pdfpagelabels,
	backref = {false},									% No index backreference
	colorlinks = {true},                  				% Color links in a PDF
	hypertexnames = {true},               				% no failures "same page(i)"
	bookmarksopen = {true},               				% opens the bar on the left side
	bookmarksopenlevel = {0},             				% depth of opened bookmarks
	pdftitle = {Pflichtenheft},	   						% PDF-property
	pdfauthor = {schma5},     					  		% PDF-property
	pdfsubject = {LaTeX Template},        				% PDF-property
	linkcolor = {linkcolor},              				% Color of Links
	citecolor = {linkcolor},           				    % Color of Cite-Links
	urlcolor = {linkcolor},               				% Color of URLs
]{hyperref}
%---------------------------------------------------------------------------

% Set up page dimension
%---------------------------------------------------------------------------
\usepackage{geometry}
\geometry{
	a4paper,
	left=28mm,
	right=15mm,
	top=30mm,
	headheight=20mm,
	headsep=10mm,
	textheight=242mm,
	footskip=15mm
}
%---------------------------------------------------------------------------

% Makeindex Package
%---------------------------------------------------------------------------
\usepackage{makeidx}                         		% To produce index
\makeindex                                    		% Index-Initialisation
%---------------------------------------------------------------------------

% Glossary Package
%---------------------------------------------------------------------------
\makeglossaries
%---------------------------------------------------------------------------

% Intro:
%---------------------------------------------------------------------------
\begin{document}                              		% Start Document	
\settocdepth{section}								% Set depth of toc
\pagenumbering{roman}														
%---------------------------------------------------------------------------

\providecommand{\titel}{Zephyr-Projekt}		% Titel der Projektarbeit								% Titel der Arbeit aus Datei titel.tex lesen
\providecommand{\versionnumber}{0.1}			%  Hier die aktuelle Versionsnummer eingeben
\providecommand{\versiondate}{22.10.2016}		%  Hier das Datum der aktuellen Version eingeben							% Versionsnummer und -datum aus Datei version.tex lesen

% Set up header and footer
%---------------------------------------------------------------------------
\makeatletter
\patchcmd{\@fancyhead}{\rlap}{\color{bfhgrey}\rlap}{}{}		% new color of header
\patchcmd{\@fancyfoot}{\rlap}{\color{bfhgrey}\rlap}{}{}		% new color of footer
\makeatother

\fancyhf{}													% clean all fields
\fancypagestyle{plain}{										% new definition of plain style	
	\fancyfoot[OR,EL]{\footnotesize \thepage} 				% footer right part --> page number
	\fancyfoot[OL,ER]{\footnotesize \titel, Version \versionnumber, \versiondate}	% footer even page left part 
}

\renewcommand{\chaptermark}[1]{\markboth{\thechapter.  #1}{}}
\renewcommand{\headrulewidth}{0pt}							% no header stripline
\renewcommand{\footrulewidth}{0pt} 							% no bottom stripline
\renewcommand*{\chapterheadstartvskip}{\vspace*{0cm}}

\pagestyle{plain}
%---------------------------------------------------------------------------


% Title Page and Abstract
%---------------------------------------------------------------------------
% !TeX encoding = ISO-8859-1
\begin{titlepage}

%---------------------------------------------------------------------------
\setlength{\unitlength}{1mm}
\begin{textblock}{20}[0,0](28,12)
	\includegraphics[scale=1.0]{bilder/BFH_Logo_B.png}
\end{textblock}
\color{black}

% Institution / Titel / Untertitel / Autoren / Experten:
%---------------------------------------------------------------------------
\begin{flushleft}
	
	
	
	
	
	
	
	
	
	
	
	
	
	

\vspace*{21mm}

\fontsize{26pt}{40pt}\selectfont 
\titel 				\\								% Titel aus der Datei vorspann/titel.tex lesen
\vspace{2mm}

\fontsize{16pt}{24pt}\selectfont\vspace{0.3em}
Echtzeit-OS f�r das Internet der Dinge	\\			% Untertitel eingeben
\vspace{5mm}

\fontsize{10pt}{12pt}\selectfont
\textbf{Pflichtenheft} \\							% eingeben
\vspace{7mm}

\begin{textblock}{150}(28,225)
\fontsize{10pt}{17pt}\selectfont

\begin{tabbing}
	xxxxxxxxxxxxxxx\=xxxxxxxxxxxxxxxxxxxxxxxxxxxxxxxxxxxxxxxxxxxxxxx \kill
	Studiengang:	\> Elektro- und Kommunikationstechnik			\\
	Institut:		\> Berner Fachhochschule						\\	
	Autoren:		\> Aaron Schmocker, David Wyss					\\
	Betreuer:		\> Martin Aebersold								\\
	Auftraggeber:	\> Martin Aebersold								\\
	Experten:		\> Martin Aebersold								\\
	Datum:			\> \versiondate									\\
\end{tabbing}

\end{textblock}
\end{flushleft}

\begin{textblock}{150}(28,280)
\noindent 
\color{bfhgrey}\fontsize{9pt}{10pt}\selectfont
Berner Fachhochschule | Haute �cole sp�cialis�e bernoise | Bern University of Applied Sciences
\color{black}\selectfont
\end{textblock}


\end{titlepage}

%
% ===========================================================================
% EOF
%
						% activate for Titelseite ohne Bild
%%
% Project documentation template
% ===========================================================================


\begin{titlepage}


% BFH-Logo absolute placed at (28,12) on A4 and picture (16:9 or 15cm x 8.5cm)
% Actually not a realy satisfactory solution but working.
%---------------------------------------------------------------------------
\setlength{\unitlength}{1mm}
\begin{textblock}{20}[0,0](28,12)
	\includegraphics[scale=1.0]{bilder/BFH_Logo_B.png}
\end{textblock}

\begin{textblock}{154}(28,48)
	\begin{picture}(150,2)
		\put(0,0){\color{bfhgrey}\rule{150mm}{2mm}}
	\end{picture}
\end{textblock}

\begin{textblock}{154}[0,0](28,50)
	\includegraphics[scale=1.0]{bilder/Zephyr-Project.jpg}			% Titelbild definieren
\end{textblock}

\begin{textblock}{154}(28,135)
	\begin{picture}(150,2)
		\put(0,0){\color{bfhgrey}\rule{150mm}{2mm}}
	\end{picture}
\end{textblock}
\color{black}

% Institution / Titel / Untertitel / Autoren / Experten:
%---------------------------------------------------------------------------
\begin{flushleft}

\vspace*{115mm}

\fontsize{26pt}{28pt}\selectfont 
\titel 				\\							% Titel aus der Datei vorspann/titel.tex lesen
\vspace{2mm}

\fontsize{16pt}{20pt}\selectfont\vspace{0.3em}
Echtzeit-OS f�r das Internet der Dinge 			\\							% Untertitel eingeben
\vspace{5mm}

\fontsize{10pt}{12pt}\selectfont
\textbf{Projektarbeit} \\									% eingeben
\vspace{3mm}

% Abstract (eingeben):
%---------------------------------------------------------------------------
\begin{textblock}{150}(28,190)
\fontsize{10pt}{12pt}\selectfont
Die Linux Foundation hat mit dem Projekt Zephyr mit der Entwicklung eines Echtzeit-Betriebssystems f�r das Internet der Dinge (IoT) begonnen.
Zephyr ist ein Open-Source-Betriebssystem mit dem Ziel ein solides OS f�r IoT Ger�te mit geringen Ressourcen bereitzustellen. Es nutzt eine echtzeitf�hige Kombination aus Nano- und Microkernel. 
Im Gegensatz zu einem Linux Kernel ben�tigt Zephyr nur zwischen 8 und 512 KByte an Arbeitsspeicher.  Aktuell werden folgenden Plattformen unterst�tzt: x86, ARM und ARC EM4
\end{textblock}

\begin{textblock}{150}(28,225)
\fontsize{10pt}{17pt}\selectfont
\begin{tabbing}
xxxxxxxxxxxxxxx\=xxxxxxxxxxxxxxxxxxxxxxxxxxxxxxxxxxxxxxxxxxxxxxx \kill
Studiengang:	\> Elektro- und Kommunikationstechnik			\\
Autoren:		\> Aaron Schmocker, David Wyss					\\
Betreuer:		\> Martin Aebersold								\\
Auftraggeber:	\> Martin Aebersold								\\
Experten:		\> Martin Aebersold								\\
Datum:			\> \versiondate									\\
\end{tabbing}

\end{textblock}
\end{flushleft}

\begin{textblock}{150}(28,280)
\noindent 
\color{bfhgrey}\fontsize{9pt}{10pt}\selectfont
Berner Fachhochschule | Haute �cole sp�cialis�e bernoise | Bern University of Applied Sciences
\color{black}\selectfont
\end{textblock}


\end{titlepage}

%
% ===========================================================================
% EOF
%
						% activate for Titelseite mit Bild

\cleardoubleemptypage
\setcounter{page}{1}
\cleardoublepage
\phantomsection 
%\addcontentsline{toc}{chapter}{Management Summary}
%% !TeX encoding = ISO-8859-1
\chapter*{Management Summary}
\label{chap:managementSummary}

% Abstract (eingeben):
%---------------------------------------------------------------------------

Diese Arbeit bietet einen �berblick �ber das Betriebssystem Zephyr mit Schwerpunkt auf Technologien, Protokolle und Anwendungsfragen. Das Zehphyr Beriebssystem ist speziell f�r das Internet der Dinge(IoT), welches durch die neuesten Entwicklungen in den Bereichen RFID, intelligente Sensoren, Kommunikationstechnologien und Internet-Protokolle schnell w�chst, entwickelt worden. Die Grundvoraussetzung ist, dass intelligente Sensoren direkt ohne menschliches Engagement zusammenarbeiten, um eine neue Klasse von Anwendungen zu liefern. Die aktuelle Revolution in den Bereichen Internet, Mobile und Machine-to-Machine (M2M) ist die erste Phase des IoT. In den kommenden Jahren wird erwartet, dass die IoT diverse Technologien �berbr�ckt, um neue Anwendungen durch die Verbindung von physischen Objekten zur Unterst�tzung intelligenter Entscheidungsfindung zu erm�glichen.  Die Linux Foundation hat mit dem Projekt Zephyr mit der Entwicklung eines Echtzeit-Betriebssystems f�r das Internet der Dinge (IoT) begonnen. Zephyr ist ein Open-Source-Betriebssystem mit dem Ziel ein solides OS f�r IoT Ger�te mit geringen Ressourcen bereitzustellen. Es nutzt eine echtzeitf�hige Kombination aus Nano- und Microkernel. Im Gegensatz zu einem Linux Kernel ben�tigt Zephyr nur zwischen 8 und 512 KByte an Arbeitsspeicher.  Aktuell werden folgenden Plattformen unterst�tzt: x86, ARM und ARC.

Dieses Arbeit beginnt mit einem horizontalen �berblick �ber die Zehphyr Beriebssystem. Anschliessend wird detailiert der Aufbau und die Ziele des Betriebsysstems erl�utert und einen �berblick �ber einige technische Details,wie z.B. Technologien, Protokolle und Anwendungen erm�glicht. Im Vergleich zu anderen Berichten, sofern diese schon vorliegen in diesem Bereich, ist es unser Ziel, eine umfassendere Zusammenfassung der wichtigsten Protokolle und Anwendungsprobleme zu bieten, damit Forscher und Anwendungsentwickler einen schnellen �berblick �ber die gew�nschten Funktionalit�ten erhalten. Des Weiteren bieten wir auch einen �berblick �ber einige andere RTOS an, um die Vor- und Nachteile des Zephyr Betriebssystems besser auf zu zeigen. Ausserdem schrieben wir eine Anleitung wie genau mau nund unsere verwendetes Noridc Eval-Board in betrieb nehmen kann und wie unsere Applikation aufgebauen ist und wie man sie bedient. Das Ziel der Arbeit
war, mit dem neuen Betreibsystem Zephyr Erfahrungen zusammeln. Evaluieren, welches Betriebssystem sich am besten f�r das Internet der Dinge (IoT) und die passenden Sensorknoten eignet, um diese Erkenntnisse in sp�teren Projekten der Berner Fachhochschule einsetzen zu k�nnen.

 


\cleardoubleemptypage
%---------------------------------------------------------------------------

% Table of contents
%---------------------------------------------------------------------------
\tableofcontents
% Versionenkontrolle :
% -----------------------------------------------

\color{black}
%\begin{huge}
\chapter*{Versionen}
%\end{huge}
\vspace{10mm}

\fontsize{10pt}{18pt}\selectfont
\begin{tabbing}
xxxxxxxxxxx\=xxxxxxxxxxxxxxx\=xxxxxxxxxxxxxx\=xxxxxxxxxxxxxxxxxxxxxxxxxxxxxxxxxxxxxxxxxxxxxxx \kill
Version	\> Datum	\> Status		\> Bemerkungen		\\
0.1	\> 15.09.2016	\> Entwurf		\> Erster Entwurf	\\	
	
\end{tabbing}



\cleardoublepage
%---------------------------------------------------------------------------

% Main part:
%---------------------------------------------------------------------------
\pagenumbering{arabic}

% !TeX encoding = ISO-8859-1
\chapter{Management Summary}
\label{chap:managementsummary}

Die Linux Foundation hat mit dem Projekt Zephyr mit der Entwicklung eines
Echtzeit-Betriebssystems f�r das Internet der Dinge (IoT) begonnen.
Zephyr ist ein Open-Source-Betriebssystem mit dem Ziel ein solides OS f�r IoT Ger�te
mit geringen Ressourcen bereitzustellen. Es nutzt eine echtzeitf�hige Kombination aus
Nano- und Microkernel.

Im Gegensatz zu einem Linux Kernel ben�tigt Zephyr nur zwischen 8 und 512 KByte an
Arbeitsspeicher. Aktuell werden folgenden Plattformen unterst�tzt: x86, ARM und ARC
EM4.


\vspace{7mm}
\begin{figure}[h]
	\centering
	\includegraphics[width=0.7\linewidth]{bilder/zephyr_components.jpg}
	\caption{Komponenten und �bersicht �ber das Zephyr RTOS}
	\label{fig:components}
\end{figure}
\vspace{7mm}
% !TeX encoding = ISO-8859-1
\chapter{Einleitung}
\label{chap:einleitung}

\begin{figure}[h]
	\centering
	\includegraphics[width=0.7\linewidth]{bilder/zephyr_components.jpg}
	\caption{Komponenten und �bersicht �ber das Zephyr RTOS}
	\label{fig:components}
\end{figure}

In dieser Projektarbeit soll zuerst ein Vergleich der Eigenschaften �hnlicher
Betriebssysteme wie FreeRTOS, RIOT, Kontiki, usw. vor allem bez�glich der Unterst�tzung der
verschiedenen Netzwerkprotokolle gemacht werden. Im Fokus steht dabei das n
Zu Demonstrationszwecken soll am Schluss mit dem ausgew�hlten Board eine Demo-App entwickelt werden.

Die Ziele sind:

\begin{itemize}
	\item Einarbeitung in das Betriebssystem Zephyr.
	\item Erstellen einer Vergleichstabelle mit den wichtigsten Eigenschaften der verschiedenen
	RTOS.
	\item Evaluation geeigneter Boards f�r das Zephyr Betriebssystem f�r eine IoT
	\item Entwicklung einer Demonstrationsapplikation, basierend auf dem ausgew�hlten Board
\end{itemize}
\include{kapitel/konzept}
\chapter{Vorgehen}
\label{chap:vorgehen}
\include{kapitel/hardware}
\include{kapitel/software}
\include{kapitel/systemtest}
\include{kapitel/interpretation}
\include{kapitel/schlussfolgerung}


%---------------------------------------------------------------------------

% Selbst�ndigkeitserkl�rung
%---------------------------------------------------------------------------
%\cleardoublepage
%\phantomsection 
%\addcontentsline{toc}{chapter}{Selbst�ndigkeitserkl�rung}
%\include{vorspann/selbstaendigkeitserklaerung}
%---------------------------------------------------------------------------

% Glossary
%---------------------------------------------------------------------------
\phantomsection 
\addcontentsline{toc}{chapter}{Glossar}
% !TeX encoding = ISO-8859-1
\chapter*{Glossar}
\label{chap:Glossar}
% \ac{} normal erste verwendung in langform, ansonsten in kurzform
% \acs{} erzwingt kurzform
% \acl{} erzwingt langform
% \acp{} plural form
\begin{acronym}[HTTPS] %[ ] l�ngste Abk�rzung f�r einr�cken
	\acro{API}{Application Programming Interface}
	\acro{CPU}{Central Processing Unit}
	\acro{RTOS}{Realtime Operating Systems}
	\acro{HTTPS}{Hypertext Transfer Protocol Secure}
	\acro{BLE}{Bluetooth Low-Energy}
	\acro{NFC}{Near Field Communication}
	\acro{IoT}{Internet der Dinge}
\end{acronym}
%\printglossary
%---------------------------------------------------------------------------
% Bibliography
%---------------------------------------------------------------------------
\phantomsection 
\addcontentsline{toc}{chapter}{Literaturverzeichnis}
%\bibliographystyle{IEEEtranS}
%\bibliography{datenbanken/bibliography}{}
\printbibliography 
%---------------------------------------------------------------------------

% Listings
%---------------------------------------------------------------------------
%\cleardoublepage
%\phantomsection 
%\addcontentsline{toc}{chapter}{Abbildungsverzeichnis}
%\listoffigures
%\cleardoublepage
%\phantomsection 
%\addcontentsline{toc}{chapter}{Tabellenverzeichnis}
%\listoftables
%---------------------------------------------------------------------------

% Index
%---------------------------------------------------------------------------
%\cleardoublepage
%\phantomsection 
%\addcontentsline{toc}{chapter}{Stichwortverzeichnis}
%\renewcommand{\indexname}{Stichwortverzeichnis}
%\printindex
%---------------------------------------------------------------------------

% Attachment:
%---------------------------------------------------------------------------
%\appendix
%\settocdepth{section}
%% !TeX encoding = ISO-8859-1
\chapter{Beliebiger Anhang}
\label{chap:bel_anhang}
%\include{anhang/beispielanhangB}
%% !TeX encoding = ISO-8859-1
\chapter{Inhalt der CD-ROM}
\label{chap:Inhalt_CDROM}

Inhaltsverzeichnis der beiliegenden CD-ROM, ev. Verzeichnisbaum, etc.
%---------------------------------------------------------------------------

%---------------------------------------------------------------------------
\end{document}

