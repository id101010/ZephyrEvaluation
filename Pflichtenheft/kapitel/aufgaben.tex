% !TeX encoding = ISO-8859-1
\chapter{Definition der Aufgaben}
\label{chap:aufgaben}

\section{Aufgabenbeschreibung}
Im Rahmen der Projektarbeit sollen die M�glichkeiten des Zephyr-OS im Hinblick auf Anwendungen im Bereich \ac{IoT} abgekl�rt werden. Besonderer Fokus gilt hierbei dem Vergleich von Zephyr zu bereits bestehenden \ac{RTOS}. Dieser Vergleich soll tabellarisch ersichtlich gemacht werden. Weiter soll im Rahmen der Arbeit eine Demonstratinsapplikation entwickelt werden, auch mit dem Fokus auf das \ac{IoT}.

\section{Anforderungen}

\begin{itemize}
    \item Minimale Anforderungen
    \subitem Aufsetzen der Entwicklungsumgebung
    \subitem Einrichten eines Gitrepositories f�r die Arbeit und Dokumentation
    \subitem Inbetriebnahme der GNU ARM Toolchain unter Eclipse Neon
    \subitem Installation der SDK und Libraries f�r das Nordic nRF52-DK
    \subitem Evaluation der F�higkeiten von Zephyr als RTOS f�r das Internet der Dinge
    \subsubitem Vergleich von Kommunikationsprotokollen mit bestehenden \ac{RTOS}
    \subsubitem Verlgeich der Codebasis mit bestehenden \ac{RTOS}
    \subsubitem Vergleich der Leistungsf�higkeit und Binarygr�sse mit bestehenden \ac{RTOS}
    \subitem Entwicklung einer Demonstrationsapplikation
    \subitem Mit Fokus auf Anwendungen rund um das Internet der Dinge
    \subitem Nutzen von \ac{BLE}
\end{itemize}

\begin{itemize}
    \item Optionale Erweiterungen
    \subitem Portierung des Zephyr-Kernels auf ESP8266
    \subitem Entwicklung einer Demoapplikation im Bereich Wireless-Lan
\end{itemize}
