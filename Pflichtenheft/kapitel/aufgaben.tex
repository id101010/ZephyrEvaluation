% !TeX encoding = ISO-8859-1
\chapter{Definition der Aufgaben}
\label{chap:aufgaben}

\section{Aufgabenbeschreibung}
Im Rahmen der Projektarbeit sollen die M�glichkeiten des Zephyr-OS im Hinblick auf Anwendungen im Bereich \ac{IoT} abgekl�rt werden. Besonderer Fokus gilt hierbei dem Vergleich von Zephyr zu bereits bestehenden \ac{RTOS}. Dieser Vergleich soll tabellarisch ersichtlich gemacht werden. Weiter soll im Rahmen der Arbeit eine Demonstratinsapplikation entwickelt werden, auch mit dem Fokus auf das \ac{IoT}.

\section{Anforderungen}

%Folgende Aufgaben sollen im Rahmen der Projektarbeit realisiert werden:

\begin{table}[H]
	\renewcommand{\arraystretch}{1.5}
	\begin{tabular}{|l|}
		\hline
		\textbf{Minimale Anforderung}\\
		\hline
		- Aufsetzen der Entwicklungsumgebung\\
		\hspace{10pt}- Aufsetzen einer Ubuntu-VM\\
		\hspace{10pt}- Einrichten eines Gitrepositories f�r die Arbeit und Dokumentation\\
		\hspace{10pt}- Inbetriebnahme der GNU ARM Toolchain unter Eclipse Neon\\
		\hspace{10pt}- Installation der SDK und Libraries f�r das Nordic nRF52-DK\\
		- Evaluation der F�higkeiten von Zephyr als RTOS f�r das Internet der Dinge\\
		\hspace{10pt}- Vergleich von Kommunikationsprotokollen mit bestehenden \ac{RTOS}\\
		\hspace{10pt}- Verlgeich der Codebasis mit bestehenden \ac{RTOS}\\
		\hspace{10pt}- Vergleich der Leistungsf�higkeit und Binarygr�sse mit bestehenden \ac{RTOS}\\	
		- Entwicklung einer Demonstrationsapplikation\\
		\hspace{10pt}- Mit Fokus auf Anwendungen rund um das Internet der Dinge\\
		\hspace{10pt}- Nutzen von \ac{BLE}\\
		\hspace{10pt}- Nutzen des nRF52-DK\\
		\hline
		\hline
		\textbf{Optionale Erweiterungen}\\
		\hline
		- Portierung des Zephyr-Kernels auf ESP8266\\
		- Entwicklung einer Demoapplikation im Bereich Wireless-Lan\\
		\hline
	\end{tabular}
\end{table}
