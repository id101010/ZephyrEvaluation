\chapter{Management Summary}
\label{chap:managementsummary}

Die Linux Foundation hat mit dem Projekt Zephyr mit der Entwicklung eines
Echtzeit-Betriebssystems für das Internet der Dinge (IoT) begonnen.
Zephyr ist ein Open-Source-Betriebssystem mit dem Ziel ein solides OS für IoT Geräte
mit geringen Ressourcen bereitzustellen. Es nutzt eine echtzeitfähige Kombination aus
Nano- und Microkernel.
Im Gegensatz zu einem Linux Kernel benötigt Zephyr nur zwischen 8 und 512 KByte an
Arbeitsspeicher. Aktuell werden folgenden Plattformen unterstützt: x86, ARM und ARC
EM4