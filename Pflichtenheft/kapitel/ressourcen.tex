\chapter{Ressourcen, Infrastruktur und Betriebsmittel}
\label{chap:ressourcen}


\section{Räumlichkeiten}

Für die Arbeit steht ein Arbeitsplatz im Raum T208 zur Verfügung.

\section{Software und Betriebsysteme}

Da es sich bei Zephyr um ein RTOS handelt, wird die Demoapplikation in C geschrieben. Zur Entwicklung und Dokumentation wird aussschliesslich auf offene Software gesetzt.

\begin{itemize}
	\item Ubuntu und Arch Linux
	\item Zephyr und Zephyr SDK
	\item GIT
	\item Latex
	\item LibreOffice
\end{itemize}

\section{Hardware}

Folgende Hardware wird für das Projekt zur Verfügung gestellt:

\begin{itemize}
	\item nRF52 Development Kit von Nordic
	\item ST-Link/v2
	\item Hardware und Computer des Raumes T208:
\end{itemize}

\section{Entwicklungs und Testwerkzeuge}

\begin{itemize}
	\item Eclipse Neon mit Gnu ARM Plugin
	\item SDK und Libraries von Nordic Semiconductors für das NRF52 Evalboard
	\item SEGGER J-Link
\end{itemize}

\section{Dokumentation}

Es wird eine Projektdokumentation erwartet, welche die Entwurfsphase sowie die Realisierung und die gemachten
Tests aufzeigt. Alle Dokumente werden auf einem Git-Repository abgelegt. Zur Dokumentation wird auf Latex gesetzt.