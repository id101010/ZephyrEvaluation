 % !TeX encoding = ISO-8859-1
 \chapter{NuttX Real-Overview}
 \label{chap:overview}
 
 \section{�bersicht �ber NuttX Real}
 
 
 NuttX Echtzeitbetriebssystem
 NuttX ist ein Echtzeitbetriebssystem (RTOS) mit einem Schwerpunkt auf Normenkonformit�t und geringem Platzbedarf. Skalierbar von 8-Bit- bis 32-Bit-Mikrocontroller-Umgebungen sind die prim�ren Normen in NuttX die Posix- und ANSI-Standards. Zus�tzliche Standard-APIs von Unix und anderen g�ngigen RTOS (wie VxWorks) werden f�r die Funktionalit�t, die unter diesen Standards nicht verf�gbar ist, oder f�r Funktionalit�ten, die f�r tief eingebettete Umgebungen (wie zum Beispiel fork ()) nicht geeignet sind, �bernommen.
 NuttX wurde zuerst 2007 von Gregory Nutt unter der permissiven BSD Lizenz ver�ffentlicht.
 
 \vspace{5mm}
 \section{Ziele}
 \vspace{5mm}
 \section{Aubau}
 \vspace{5mm}
 \section{Unterst�tztung}