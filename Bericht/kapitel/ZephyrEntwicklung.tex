% !TeX encoding = ISO-8859-1
\chapter{Applikationsentwicklung mit Zephyr}
\label{chap:zephyrdevel}

\section{�bersicht}

Das Buildsystem des Zephyr-Kernels basiert auf Kbuild. Kuild ist das Sytem welches f�r den Linux Kernel entwickelt wurde.

Um eine Applikation zu erstellen ist ein Konfigurationsfile f�r den Kernel notwendig. Das Buildsystem kompiliert dann sowohl den Kernel als auch die Applikation und erstellt ein einziges Binary daraus.

Im folgenden wir unterschieden zwischen dem Kernel-Ordner und dem Applikations-Ordner. Der Kernel-Ordner enth�lt den vollst�ndigen Quellcode des Kernels, die Standardkonfiguration und die Build Definitionen f�r den Kernel.

Der Applikationsordner enth�lt den ganzen Quellcode und spezifische Konfigurationen f�r den Kernel. Im Quellcode der Applikation wird lediglich auf den Kernel gelinkt. Wenn keine neuen Treiber entwickelt werden sollen, beschr�nkt sich die Applikationsentwicklung ausschliesslich auf die Konfigurationen und den Quellcode im Applikationsverzeichnis. Der Kernel selbst muss nicht ver�ndert werden, da dies vom Buildsystem erledigt wird.

\section{Applikationsstruktur}

Das Minimum einer Applikationsstruktur sieht folgendermassen aus:

\begin{itemize}
	\item \textbf{Quelltext der Applikation} Applikationen werden typischerweise in C oder Assembler geschrieben und liegen im Ordner src.
	\item \textbf{Kernel Konfigurationsfiles} Eine Zephyrapplikation stellt dem Buildsystem ein Konfigurationsfile *.conf zur Verf�gung. Fehlt diese Datei, wird die Standardkonfiguration verwendet.
	\item \textbf{Makefile} Diese Datei teilt dem Buildsystem lediglich mit wo die oben genannten Dateien zu finden sind und um was f�r ein Board es sich beim Target handelt. Es handelt sich dabei nicht um ein Makefile im klassischen Sinne.
\end{itemize}

Wurde die Applikation nach obigem Standard definiert, kann sie mit einem einzigen Aufruf vom \textbf{make} gebaut werden. Das Kompilat findet sich in einem eigens daf�r erstellten Unterordner namens \textbf{outdir/BOARD}. Vom Buildsystem werden folgende Dateien generiert:

\begin{itemize}
	\item Das \textbf{.config} File enth�lt die Einstellungen welche vom Buildsystem f�r das Erstellen der Applikation benutzt wurden.
	\item Object-Files \textbf{.o und .a} welche Kompilat aus Kernel und Applikation enthalten.
	\item Das \textbf{zephyr.elf} Binary, welches Kernel- und Applikationscode in sich vereint.
\end{itemize}  

\section{Bedienung des Build-Systems}

Die Verwendung des Buildsystems gestaltet sich denkbar einfach. Es muss lediglich das zentrale Makefile ins Projektmakefile inkludiert werden, das Programm Make erledigt dann den Rest. Folgende Zeile muss dem Makefile hinzugef�gt werden:

\begin{lstlisting}[style=BashInputStyle]
$ include $(ZEPHYR_BASE)/Makefile.inc
\end{lstlisting}

Weiter erwartet das Buildsystem einige Umgebungsvariablen, die jedoch im Normalfall automatisch gesetzt werden. Der Vollst�ndigkeit halber folgt im Anschluss eine Liste.

\begin{itemize}
	\item \textbf{ZEPHY-RBASE} Hier steht der Pfad zum Ordner welcher den vollst�ndigen Kernel Sourcecode enth�lt. Diese Variable wird �ber das Sourcen des zephyr-env.sh Skriptes gesetzt, wie in Kapitel 2.5.3 Beschrieben.
	
	\item \textbf{PROJECT-BASE} Hier steht der Pfad zum Applikationsordner. Diese Variable wird durch das Makefile gesetzt.
	
	\item \textbf{SOURCE-DIR} Hier steht der Pfad zum Quelltext der Applikation, per Standard zeigt diese Variable auf den Unterordner src.
	
	\item \textbf{BOARD} Diese Variable gibt an f�r welches Board die Applikation kompiliert werden soll.
	
	\item \textbf{CONF-FILE} Diese Variable gibt an welche speziellen Kernelkonfigurationen vorgenommen werden sollen.
	
	\item \textbf{O} Hier befindet sich der Ordner in welchem sich das kompilierte Binary befinden wird. Standardm�ssig wird dies der Unterordner outdir sein.
\end{itemize}  

\section{Konfiguration des Makefiles einer Applikation}

\section{Konfiguration des Kernels einer Applikation}

\section{Debugging mit GDB}

\section{nrf52dk-zephyr-tool.sh}

Das Makefile der Zephyr-Build-Tools erlaubt es mittels \textbf{make flash} Software auf Boards direkt herunterzuladen. Jedoch muss beim nrf52 Development-Kit auf propriet�re Software von Nordic zur�ckgegriffen werden. Da diese Software nicht Quelloffen ist, kann sie nicht mit den Build-Tools ausgeliefert werden.
Um die Entwicklung von Zephyrapplikationen dennoch zu vereinfachen ist im Rahmen der Projektarbeit ein kleines Shellskript entstanden welches die grundlegenden Operationen wie das erstellen eines Projektes mit Minimalkonfiguration, das Kompilieren einer Applikation oder das Herunterladen eines Binarys auf das nrf52 Development-Kit erlaubt. Die Bedienung des Shellskripts h�lt sich an die g�ngigen Unix Standards.

\begin{lstlisting}[style=BashInputStyle]
$ ./nrf52dk_zephyr_tool.sh help

Usage: nrf52dk_zephyr_tool.sh [options]

This script handles the creation and compilation of zephyr applications for the nrf52dk.

OPTIONS:
help                 Show this message
flash  ProjectPath   Flashes the previously built project
build  ProjectPath   Builds a zephyr application
create ProjectPath   Creates a new project folder with standard configuration

\end{lstlisting}


