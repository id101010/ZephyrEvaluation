% !TeX encoding = ISO-8859-1
\chapter{nRF52DK}
\label{chap:nrf52dk}

Um die Funktion aller Tools und der SDK zu gew�hrleisten sollten folgende Installationen ausschliesslich in der hier beschriebenen Reihenfolge ausgef�hrt werden.

\section{Installation der GNU ARM Embedded Toolchain}

Um die Software auf dem Hostsystem f�r das nRF52DK kompilieren zu k�nnen ben�tigen wir eine sogenannte crosscompiler-toolchain. F�r ARM basierte Controller ist hier die erste Wahl die gnu-arm-none-eabi Toolchain. Sie l�sst sich von folgender Seite beziehen:

\url{https://launchpad.net/gcc-arm-embedded/+download}

Es gilt darauf zu achten die neueste Version zu verwenden und sich zu notieren welche Version heruntergeladen wurde, da diese sp�ter im Makefile angegeben werden muss. Die Software sollte im Verzeichnis /opt installiert werden. Die Binaries sollten mit den n�tigen Rechten ausgestattet werden. Weiter empfiehlt es sich symbolische Links zu generieren um die Programme systemweit sichtbar zu machen.

Die Befehle dazu lauten:

\begin{lstlisting}[style=BashInputStyle]
    tar -xf gcc-arm-none-eabi-linux.tar.bz2 /opt/gcc-arm-none-eabi-tools
    ln -s /opt/gcc-arm-none-eabi-tools/bin/* /usr/local/bin/
\end{lstlisting}

Unter ArchLinux l�sst sich die Software auch aus den offiziellen Paketquellen installieren, die Befehle dazu lauten:

\begin{lstlisting}[style=BashInputStyle]
    pacman -S community/arm-none-eabi-binutils
    pacman -S community/arm-none-eabi-newlib
    pacman -S community/arm-none-eabi-gcc
    pacman -S community/arm-none-eabi-gdb
\end{lstlisting}


\section{Installation der nordic SDK}

Nordic Semiconductor verf�gt �ber ein Software Archiv von welchem alle ben�tigte Software bezogen werden kann. Das Archiv befindet sich unter folgendem Link:

\url{https://www.nordicsemi.com/eng/Products/Bluetooth-low-energy/nRF52-DK#Downloads}

Um mit dem nRF52 Entwicklungskit arbeiten zu k�nnen ben�tigen wir folgende Softwarepakete:

\begin{enumerate}
    \item nRF5 SDK Zip File
    \item nRF5x-Command-Line-Tools-Linux64
\end{enumerate}

Im ersten Paket befinden sich alle Files welche zur Entwicklung von Software auf dem nRF52dk notwendig sind. Darunter Makefiles, Linkerskripte und auch Beispielcode.
Im zweiten Paket befinden sich die Command-Line-Tools, welche es uns erm�glichen �ber die Kommandozeile mit dem nRF52 Entwicklungskit zu kommunizieren. Das wichtigste Tool ist nrfjprog welches kompilierte Programme auf das Board hochladen kann.

Die SDK sowie die Command-Line-Tools sollten auf dem System unter dem Ordner /opt installiert werden und mit den n�tigen Berechtigungen versehen werden. Ebenfalls sollten die Programme in einen Ordner gelinkt werden welcher im Systempfad angegeben ist.

\clearpage

Die Befehle dazu lauten:
\begin{lstlisting}[style=BashInputStyle]
    tar -xf nRF5x-Command-Line-Tools-Linux64.tar /opt/
    ln -s /opt/nrfjprog/nrfjprog /usr/bin/nrfjprog
    ln -s /opt/mergehex/mergehex /usr/bin/mergehex
\end{lstlisting}

Unter Arch Linux lassen sich die Command Line Tools auch �ber das Arch User Repository beziehen. Dazu kann man den Pacman Wrapper "yaourt" verwenden:

\begin{lstlisting}[style=BashInputStyle]
    yaourt -S aur/nrf5x-command-line-tools 
\end{lstlisting}

\section{Installation des SEGGER JLink Debuggers}

Die Software von SEGGER, welche f�rs Debuggen gebraucht wird, findet man unter folgendem Link:

\url{https://www.segger.com/downloads/jlink}

Das Paket "J-Link Software and Documentation Pack" enth�lt verschiedene Programme. Die Wichtigsten davon sind der JLinkCommander und der JLinkGDBServer. �ber den Commander l�sst sich die CPU des nRF52 vollst�ndig via JTAG oder SWD Schnittstelle kontrollieren. Der JLinkGDBServer stellt auf dem Localhost einen Socket unter Port 2331 zur Verf�gung, auf welchen man sich mit GDB verbinden kann.

Die JLink Toolchain sollte auf dem System unter dem Ordner /opt installiert werden und mit den n�tigen Berechtigungen versehen werden. Die Programme sollten mittels symbolischen Links in einen Ordner gelinkt werden welcher im Systempfad angegeben ist.

\begin{lstlisting}[style=BashInputStyle]
    mkdir /opt/SEGGER/JLink
    tar -xzf JLink_Linux_{VERSION}.tar.gz /opt/SEGGER/JLink
    ln -s /opt/SEGGER/JLink/JLinkExe /usr/bin/JLinkExe
    ln -s /opt/SEGGER/JLink/JLinkGDBServer /usr/bin/JLinkGDBServer
\end{lstlisting}

SEGGER bietet auch ein Debianpaket an. Dieses kann auf Systemen welche �ber den Paketmanager DPKG verf�gen mittels \verb|dpkg -i JLink_linux.deb| installiert werden. Dadurch entf�llt obiger Installationsaufwand.\newline

Unter ArchLinux l�sst sich die Software auch aus dem Arch-User-Repository installieren, der Befehl dazu lautet:
\begin{lstlisting}[style=BashInputStyle]
    yaourt -S aur/jlink-software-and-documentation
\end{lstlisting}

\section{Verwendung des JLinkGDBServers}

\begin{lstlisting}[style=BashInputStyle]
    JLinkGDBServer -Device nRF52832_xxAA -If SWD -Speed 4000 -Autoconnect 1
\end{lstlisting}

\section{Verwendung von nrfjprog}

\begin{lstlisting}[style=BashInputStyle]
    nrfjprog --eraseall -f nrf52
    nrfjprog --program outdir/nrf52_pca10040/zephyr.hex -f nrf52
    nrfjprog --reset -f nrf52
\end{lstlisting}

\section{Vorteile}
\section{Technische Beschreibung des Boards}
\section{Bluetooth Low Energy}
\section{Analog to Digital Converter}
\section{I2C Bus}