% !TeX encoding = ISO-8859-1
\chapter{Fazit}
\label{chap:fazit}

\section{Vergleich Ist/Soll}

Wir sind mit dem erreichten Resultat zufrieden. Unsere Arbeit bietet einen horizontalen �berblick �ber das Zephyr-Betriebssystem. Ausserdem konnte ein detaillierter Beschrieb des Aufbaus und der Ziele des Betriebssystems erl�utert werden. Zudem wird ein �berblick �ber einige technische Details, wie z.B. Technologien, Protokolle und Anwendungen gew�hrt. Im Vergleich zu anderen Berichten, sofern diese schon vorliegen in diesem Bereich, konnten wir unser Ziel, eine umfassendere Zusammenfassung der wichtigsten Protokolle und Anwendungsprobleme bieten, damit Forscher und Anwendungsentwickler einen schnellen �berblick �ber die gew�nschten Funktionalit�ten erhalten. 

Ein weiteres Ziel, das wir uns f�r die Projektarbeit gestellt haben, ist die Entwicklung einer Demonstrationsapplikation f�r Zephyr, welche als Beispielprojekt f�r Neuentwicklungen dienen kann. Im Verlaufe der Projektarbeit musste das Pflichtenheft dieser Demoapp leider abgespeckt werden. Das Zephyr-OS war zur Zeit der Durchf�hrung der Projektarbeit noch unter aktiver Entwicklung. Auch die Unterst�tzung des nRF52 Development Kits beschr�nkte sich gr�sstenteils auf Bluetooth und Wireless. Die Dokumentation aller weiteren Funktionalit�ten f�r das Board war noch nicht vorhanden, was Entwicklungsarbeiten enorm erschwerte. Die Ursache ist damit zu begr�nden, dass das Board wegen der propriet�ren Toolchain vom Buildsystem noch nicht vollst�ndig unterst�tzt ist. Das bedeutet, dass zur Entwicklung Tools wie die Eclipse IDE nur beschr�nkt einsetzbar sind.
Im Rahmen der Projektarbeit haben wir daher ein Shellscript geschrieben, welche die oben genannten Funktionalit�ten mit bringt. Eine Anpassung des Makefiles des Zephyr Build-Systems w�re w�nschenswert und w�rde den Support der Produktepalette von Nordic ganz sicher vorantreiben. Dies h�tte jedoch den von uns gesteckten Rahmen der Arbeit gesprengt, daher haben wir uns auf die Entwicklung des Shellscripts begrenzt.
Als Demonstrationsapplikation dient momentan das Eddystone Beacon Example des Zephyrprojektes. Das Beispielprogramm wurde jedoch f�rs nRF52 Board angepasst und in ein eigenes Projekt umgewandelt. Weiter entwickelten wir eine ''Hello World'' Applikation, welche vom nrf52dk-zephyr-tool beim Anlegen eines neuen Projektes generiert wird. Diese Applikation zeigt die Verwendung der GPIOs des Boards. 
Die gesammelten Erfahrungen k�nnen in sp�teren Projekten der Berner Fachhochschule bestimmt gut gebraucht werden. Als Schlussbetrachtung kann gesagt werden, dass das Projekt eine gewisse Herausforderung war. Viele Teile des Zephyr-Projektes sind noch nicht dokumentiert und es bedurfte bis zu einem gewissen Grad ''Reverse Engineering'' und ''Trial-And-Error'', um beispielsweise die richtigen Kernelkonfigurationen zu finden, damit die Software auf unserem Board l�uft. Nichtsdestotrotz war es ein sehr interessantes und abwechslungsreiches Projekt. 

Wir m�chten uns hiermit bei unserem Betreuer Herr Martin Aebersold f�r die gute Unterst�tzung bedanken.

\section{W�nschenswerte Erweiterungen}

W�nschenswert w�re sicher die Erweiterung des Makefiles, um das nRF52 Board besser zu unterst�tzen. Weiter sollte man sich vertiefter mit der Entwicklung von Sensortreibern auseinandersetzen. Zephyrs Erfolg h�ngt schlussendlich davon ab, wie gut die zur Zeit popul�ren Sensoren und Ger�te unterst�tzt werden.   
Je mehr Support, desto weiter wird sich das RTOS verbreiten k�nnen. Hier sehen wir auch grosse Chancen f�r die Open-Source Entwicklung. Nachfolgende Projekte k�nnten sich beispielsweise mit der Entwicklung eines Treibers besch�ftigen. 

Auch die Portierung des Zephyr Kernels auf eine andere Plattform, wie beispielsweise das ESP8266, w�re ein interessantes Thema f�r ein Folgeprojekt. 

