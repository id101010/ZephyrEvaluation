% !TeX encoding = ISO-8859-1
\chapter{Fazit}
\label{chap:fazit}

\section{Vergleich Ist/Soll}

Wir sind mit dem erreichten Resultat zufrieden. Unsere Arbeit bietet einen horizontalen �berblick �ber das Zephyr-Betriebssystem. Ausserdem konnte eine detaillierter Beschrieb des Aufbaus und der Ziele des Betriebssystem erl�utert werden. Zudem wird ein �berblick �ber einige technische Details, wie z.B. Technologien, Protokolle und Anwendungen gew�hrt. Im Vergleich zu anderen Berichten, sofern diese schon vorliegen in diesem Bereich, konnten wir unser Ziel, eine umfassendere Zusammenfassung der wichtigsten Protokolle und Anwendungsprobleme bieten, damit Forscher und Anwendungsentwickler einen schnellen �berblick �ber die gew�nschten Funktionalit�ten erhalten. 


Des Weiteren bieten wir auch einen �berblick �ber einige andere RTOS an, um die Vor- und Nachteile des Zephyr Betriebssystems besser auf zu zeigen. Ausserdem schrieben wir eine Anleitung. Wie genau man unser verwendetes Noridc Eval-Board in Betrieb nehmen kann. Zu dieser Anleitung wurde zus�tzlich ein Skript geschrieben, welches des Einstieg erleichtern soll. Ausserdem wurde weiteres Skript geschrieben, welches ein Beispielprojekt f�r das Zephyr-Betriebssystem generiert, damit ein Anwender gerade sein erstes Programm f�r Zephyr entwickeln kann.

Unsere Applikation, welche wir im Verlaufe unserer Projektarbeit entwickelt haben, l�uft auf dem nRF52Dk Board und sendet via Bluetooth ein ''Eddystone Beacon''. Dieses Beacon kann man mittels nrF Connect App auf einem Smartphone abgerufen werden. Wir konnten somit Erfahrungen mit dem Betreibsystem Zephyr sammeln. Diese Erfahrungen k�nnen nun in sp�teren Projekten der Berner Fachhochschule einsetzt werden. Als Schlussbetrachtung muss gesagt werden, dass die gestellte Aufgabe eine echte Herausforderung war. Da viele Teile des Zephyr-Projektes noch sehr schlecht Dokumentiert waren und wir viele Dinge mit reverse engineering herausfinden mussten. Nichtsdestotrotz war es ein sehr interessantes und abwechslungsreiches Projekt. Wir m�chten uns hiermit bei unseren Betreuern Dr. Werner Jenni die gute Unterst�tzung bedanken.


\section{W�nschenswerte Erweiterungen}

Was uns leider wegen Zeit mangels nicht gereicht hat, ist das Zephyr-Betriebssystem auf einen nicht unterst�tze Hardware zu portieren. Dies m�sste man, bei einer allf�lligen weiteren Arbeit nachholen, damit man das Zephyr auch auf anderen Hardware Plattformen einsetzen kann. Des Weiteren k�nnten man eine noch umfangreichere Applikation f�r andere Anwendungen schreiben.