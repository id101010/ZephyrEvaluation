% !TeX encoding = ISO-8859-1
\chapter{ZephyrOverview}
\label{chap:overview}

\section{�bersicht �ber Zephyr}

Zephyr ist laut Beschreibung der Linux-Foundation \cite{LinuxFoundation} ein Open-Source-Echtzeitbetriebssystem, speziell optimiert f�r Anwendungen im Internet der Dinge. Die Architektur basiert auf einer echtzeitf�higen Kombination von Nano- und Mikrokernel. Es wird aktuell von der Linux-Foundation in Zusammenarbeit mit den Firmen Intel, NXP und Synopsys in der Form eines Collaborative-Projects entwickelt. Dadurch soll versucht werden die bei der Linux- und Open-Source-Entwicklung verwendeten Arbeitsweisen und Ideen auch im Bereich der Industrie einzubringen.
Ziel ist es ein robustes und sicheres Betriebssystem f�r das Internet der Dinge zu schaffen. Zephyr ist vollst�ndig Open-Source steht laut Information des Newsportals Heise.de \cite{HeiseDe} unter der Apache Lizenz Version 2.0. 
Dieses Lizenzierungsmodell kommt Firmen und Unternehmen entgegen, welche den Einsatz von Open-Source-Software generell scheuen da diese oft unter der \ac{GPL} stehen. [2] Wird in Produkten Software verwendet welche unter der \ac{GPL} lizensiert ist, zwingt dieses Lizenzmodell die Firmen dazu ihre Produkte ebenfalls unter GPL zu ver�ffentlichen. Dies beinhaltet auch s�mtliche �nderungen welche vorgenommen wurden. Bei der Apache-Lizenz ist dies gem�ss Definition \cite{ApacheLicense} nicht zwingend.

Momentan unterst�tzt der Zephyr-Kernel gem�ss Angaben auf der Projektseite \cite{ZephyrProjectDocumentation} Prozessoren der Architekturen ARC, ARM-v7 aber auch x86. Dadurch ist das System auf popul�ren Plattformen wie dem Arduino 101 Board, dem Arduino Due Board, dem NXP Freedom DK lauff�hig und Intel Galileo Gen 2. Zur Kommunikation stehen unter anderem Protokolle Ipv4, Ipv6, Bluetooth 4.0, LoWPAN zur Verf�gung.

\section{Ziele}

Zephyr ist f�r den Einsatz auf Ger�ten mit geringem Speicherplatz und feststehender Hardwarekonfiguration gedacht. Darunter fallen unter anderem Steuerungen f�r Heizungs- und Beleuchtungssysteme aber auch Ger�te aus allen Bereichen des t�glichen Lebens mit Internet-Anbindung.

Das ZephyrProjekt verfolgt laut Beschreibung der Linux-Foundation \cite{LinuxFoundation} folgende Ziele:
\begin{itemize}
    \item Kleiner footprint - lauff�hig mit minimal 10kB
    \item CPU unabh�ngige Architektur
    \item Modular und Skalierbar
    \item Hoche Sicherheitsstandards
    \item Unterst�tzt von Grund auf viele unterschiedliche Boards und Kommunikationsprotokolle
    \item M�chtige Entwicklungswerkzeuge
    \item OpenSource Kernel mit Apache v2.0 Lizenz
\end{itemize}

\section{Aufbau}

Das Zephyr OS setzt auf eine Kombination von Nano- und Mikrokernel. Dadurch soll Zephyr  bereits mit nur 10 Kbyte an Speicherplatz lauff�hig sein. Das macht Zephyr besonders f�r Anwendungen auf kleinen Mikrokontrollern attraktiv. Bei einem herk�mmlichen Linux-Kernel w�re dies nicht denkbar. G�ngige Adaptionen f�r Smartphone SoCs ben�tigen laut [1] in der kleinsten Konfiguration noch bis zu 200KB RAM und rund 1MB Flash.

Der Nanokernel bietet Echtzeit-F�higkeiten. Die Zeit die der Nanokernel f�r die Abarbeitung einer Aufgabe ben�tigt ist also deterministisch. Dies unabh�ngig davon wie stark das System gerade ausgelastet ist. F�r alle Aufgaben welche keine Anforderungen an Echtzeit-Verarbeitung stellen verf�gt das System �ber einen Microkernel. 
[1] http://linuxdevices.linuxgizmos.com/my-linux-is-smaller-than-your-linux-a/

\section{Aufsetzen der SDK unter Ubuntu}

\section{Vergleich mit anderen RTOS}

\section{Sicherheitsaspekte}

\section{Portierungsaufwand}