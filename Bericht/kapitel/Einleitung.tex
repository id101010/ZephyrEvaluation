% !TeX encoding = ISO-8859-1
\chapter{Einleitung}
\label{chap:einleitung}

F�r das Modul ''Projektarbeit und System Engineering BTE551'' wird ein Projekt bearbeitet. Das Projekt wurde w�hrend der unterrichtfreien Zeit ausgeschrieben und den jeweiligen Studierenden zugewiesen. Das Ziel dieser Arbeit ist die Studierenden mit Projektarbeiten vertraut zu machen.

Im Rahmen der Projektarbeit des 5. Semesters realisiert dieses Projekt einen Vergleich zwischen dem Zephyr-Betriebssystem und �hnlichen Betriebssysteme wie FreeRTOS, RIOT, Contiki usw. Ausserdem sollen die F�higkeiten des Zephyr-ROTS in bestehender Form ausgetestet werden. Spezielle Beachtung soll den f�r das Internet der Dinge wichtigen Kommunikationsprotokollen wie BLE und Near Field Communication (NFC) getestet werden. Anschliessend sind geeignete Boards f�r das Betriebssystem Zephyr zu evaluieren. Des Weiteren sollen die zwei unterschiedlichen Kernel des Zephyr-Betriebssystems verglichen werden. Zu Demonstrationszwecken soll am Schluss mit dem ausgew�hlten Board eine kleine Demo-App entwickelt werden.

Man m�chte mit dem neuen Betreibsystem Zephyr Erfahrungen sammeln. Evaluieren, welches Betriebssystem sich am besten f�r das Internet der Dinge (IoT) und die passenden Sensorknoten eignet, um diese Erkenntnisse in sp�teren Projekten einsetzen zu k�nnen.

Mit diesem Projekt soll intern das Know-how f�r das Zephyr-Open-Source-Betriebssystem, welches ein solides OS f�r IoT Ger�te mit geringen Ressourcen bereitstellt und eine echtzeitf�hige Kombination aus Nano- und Microkernel nutzt, ausgebaut werden. Das gesammelte Wissen wird dem Fachbereich Elektrotechnik f�r den Unterricht zur Verf�gung gestellt. Eine weitere Kundenanforderung ist es, dass wir bei unserer Projektarbeit nur Open-Source Software und Softwaretools verwenden, somit m�chten wir unser Wissen auch der Open-Source-Community zur Verf�gung stellen.


