% ===========================================================================
%  ______          _           _     ______           _
%  | ___ \        (_)         | |   |___  /          | |
%  | |_/ / __ ___  _  ___  ___| |_     / /  ___ _ __ | |__  _   _ _ __
%  |  __/ '__/ _ \| |/ _ \/ __| __|   / /  / _ \ '_ \| '_ \| | | | '__|
%  | |  | | | (_) | |  __/ (__| |_  ./ /__|  __/ |_) | | | | |_| | |
%  \_|  |_|  \___/| |\___|\___|\__| \_____/\___| .__/|_| |_|\__, |_|
%                _/ |                          | |           __/ |
%               |__/                           |_|          |___/
%
%                       Main Document Project Zephyr
% ===========================================================================
%
%   Needed ArchLinux Packaes:
%
%   texlive-bin
%   texlive-core
%   texlive-fontsextra
%   texlive-bibtexextra
%   texlive-genericextra
%   texlive-latexextra
%   texlive-publishers
%

%---------------------------------------------------------------------------
\documentclass[
    a4paper,                                                % paper format
    10pt,                                                   % fontsize
    twoside,                                                % double-sided
    openright,                                              % begin new chapter on right side
    notitlepage,                                            % use no standard title page
    parskip=half,                                           % set paragraph skip to half of a line
]{scrreprt}
%---------------------------------------------------------------------------

\raggedbottom
\KOMAoptions{cleardoublepage=plain}                         % Add header and footer on blank pages


% Load Standard Packages:
%---------------------------------------------------------------------------
\usepackage[standard-baselineskips]{cmbright}
\usepackage[ngerman]{babel}                                 % german hyphenation
\usepackage[ansinew]{inputenc}                              % Windows - load extended character set (ISO 8859-1)
\usepackage[T1]{fontenc}                                    % hyphenation of words with ���
\usepackage{textcomp}                                       % additional symbols
\usepackage{ae}                                             % better resolution of Type1-Fonts
\usepackage{fancyhdr}                                       % simple manipulation of header and footer
\usepackage{etoolbox}                                       % color manipulation of header and footer
\usepackage{graphicx}                                       % integration of images
\usepackage{float}                                          % floating objects
\usepackage{caption}                                        % for captions of figureMain documents and tables
\usepackage{booktabs}                                       % package for nicer tables
\usepackage{tocvsec2}                                       % provides means of controlling the sectional numbering
\usepackage{listings}                                       % Highlight aaaall the codes!

\usepackage[]{acronym}                                      % Glossary
\usepackage[acronym,toc]{glossaries}
%---------------------------------------------------------------------------

% Load Math Packages
%---------------------------------------------------------------------------
\usepackage{amsmath}                                        % various features to facilitate writing math formulas
\usepackage{amsthm}                                         % enhanced version of latex's newtheorem
\usepackage{amsfonts}                                       % set of miscellaneous TeX fonts that augment the standard CM
\usepackage{amssymb}                                        % mathematical special characters
\usepackage{exscale}                                        % mathematical size corresponds to textsize
%---------------------------------------------------------------------------

% Package to facilitate placement of boxes at absolute positions
%---------------------------------------------------------------------------
\usepackage[absolute]{textpos}
\setlength{\TPHorizModule}{1mm}
\setlength{\TPVertModule}{1mm}
%---------------------------------------------------------------------------

% Definition of Colors
%---------------------------------------------------------------------------
\RequirePackage{color}                                      % Color (not xcolor!)
\definecolor{orange}{RGB}{244, 182, 66}
\definecolor{purple}{RGB}{188, 66, 244}
\definecolor{linkblue}{rgb}{0,0,0.8}                        % Standard
\definecolor{darkblue}{rgb}{0,0.08,0.45}                    % Dark blue
\definecolor{bfhgrey}{rgb}{0.41,0.49,0.57}                  % BFH grey
\definecolor{linkcolor}{rgb}{0,0,0.8}                       % Blue for the web- and cd-version!
\definecolor{linkcolor}{rgb}{0,0,0}                         % Black for the print-version!
\definecolor{codebackground}{RGB}{196, 218, 255}            % Codehighlight background
%---------------------------------------------------------------------------

% Codehighlightings
%---------------------------------------------------------------------------
\lstdefinestyle{BashInputStyle}{
    belowcaptionskip=1\baselineskip,
    breaklines=true,
    frame=tb,
    numbers=left,
    numberstyle=\tiny,
    linewidth=0.95\linewidth,
    xleftmargin=0.05\linewidth,
    language=bash,
    showstringspaces=false,
    basicstyle=\small\sffamily,
    keywordstyle=\bfseries\color{black},
    backgroundcolor=\color{codebackground},
}

\lstdefinestyle{customc}{
    belowcaptionskip=1\baselineskip,
    breaklines=true,
    frame=tb,
    numbers=left,
    numberstyle=\tiny,
    linewidth=0.95\linewidth,
    xleftmargin=0.05\linewidth,
    language=C,
    showstringspaces=false,
    basicstyle=\footnotesize\ttfamily,
    keywordstyle=\bfseries\color{black},
    backgroundcolor=\color{codebackground},
    commentstyle=\itshape\color{purple},
    identifierstyle=\color{blue},
    stringstyle=\color{orange},
}
%---------------------------------------------------------------------------

% Hyperref Package (Create links in a pdf)
%---------------------------------------------------------------------------
\usepackage[
    pdftex,ngerman,bookmarks,plainpages=false,pdfpagelabels,
    backref = {false},                                      % No index backreference
    colorlinks = {true},                                    % Color links in a PDF
    hypertexnames = {true},                                 % no failures "same page(i)"
    bookmarksopen = {true},                                 % opens the bar on the left side
    bookmarksopenlevel = {0},                               % depth of opened bookmarks
    pdftitle = {Projekt Zephyr},                            % PDF-property
    pdfauthor = {schma5},                                   % PDF-property
    pdfsubject = {LaTeX Bericht},                           % PDF-property
    linkcolor = {linkcolor},                                % Color of Links
    citecolor = {linkcolor},                                % Color of Cite-Links
    urlcolor = {linkcolor},                                 % Color of URLs
]{hyperref}
%---------------------------------------------------------------------------

% Set up page dimension
%---------------------------------------------------------------------------
\usepackage{geometry}
\geometry{
    a4paper,
    left=28mm,
    right=15mm,
    top=30mm,
    headheight=20mm,
    headsep=10mm,
    textheight=242mm,
    footskip=15mm
}
%---------------------------------------------------------------------------

% Makeindex Package
%---------------------------------------------------------------------------
\usepackage{makeidx}                                        % To produce index
\makeindex                                                  % Index-Initialisation
%---------------------------------------------------------------------------

% Glossary Package
%---------------------------------------------------------------------------
\makeglossaries
%---------------------------------------------------------------------------

% Intro:
%---------------------------------------------------------------------------
\begin{document}                                            % Start Document
\settocdepth{section}                                       % Set depth of toc
\pagenumbering{roman}
%---------------------------------------------------------------------------

\providecommand{\titel}{Zephyr-Projekt}		% Titel der Projektarbeit                                      % Titel der Arbeit aus Datei titel.tex lesen
\providecommand{\versionnumber}{0.1}			%  Hier die aktuelle Versionsnummer eingeben
\providecommand{\versiondate}{22.10.2016}		%  Hier das Datum der aktuellen Version eingeben                                    % Versionsnummer und -datum aus Datei version.tex lesen

% Set up header and footer
%---------------------------------------------------------------------------
\makeatletter
\patchcmd{\@fancyhead}{\rlap}{\color{bfhgrey}\rlap}{}{}     % new color of header
\patchcmd{\@fancyfoot}{\rlap}{\color{bfhgrey}\rlap}{}{}     % new color of footer
\makeatother

\fancyhf{}                                                  % clean all fields
\fancypagestyle{plain}{                                     % new definition of plain style
    \fancyfoot[OR,EL]{\footnotesize \thepage}               % footer right part --> page number
    \fancyfoot[OL,ER]{\footnotesize \titel, Version \versionnumber, \versiondate} % footer even page left part
}

\renewcommand{\chaptermark}[1]{\markboth{\thechapter. #1}{}}
\renewcommand{\headrulewidth}{0pt}                          % no header stripline
\renewcommand{\footrulewidth}{0pt}                          % no bottom stripline

\pagestyle{plain}
%---------------------------------------------------------------------------


% Title Page and Abstract
%---------------------------------------------------------------------------
%
% Project documentation template
% ===========================================================================


\begin{titlepage}


% BFH-Logo absolute placed at (28,12) on A4 and picture (16:9 or 15cm x 8.5cm)
% Actually not a realy satisfactory solution but working.
%---------------------------------------------------------------------------
\setlength{\unitlength}{1mm}
\begin{textblock}{20}[0,0](28,12)
	\includegraphics[scale=1.0]{bilder/BFH_Logo_B.png}
\end{textblock}

\begin{textblock}{154}(28,48)
	\begin{picture}(150,2)
		\put(0,0){\color{bfhgrey}\rule{150mm}{2mm}}
	\end{picture}
\end{textblock}

\begin{textblock}{154}[0,0](28,50)
	\includegraphics[scale=1.0]{bilder/Zephyr-Project.jpg}			% Titelbild definieren
\end{textblock}

\begin{textblock}{154}(28,135)
	\begin{picture}(150,2)
		\put(0,0){\color{bfhgrey}\rule{150mm}{2mm}}
	\end{picture}
\end{textblock}
\color{black}

% Institution / Titel / Untertitel / Autoren / Experten:
%---------------------------------------------------------------------------
\begin{flushleft}

\vspace*{115mm}

\fontsize{26pt}{28pt}\selectfont 
\titel 				\\							% Titel aus der Datei vorspann/titel.tex lesen
\vspace{2mm}

\fontsize{16pt}{20pt}\selectfont\vspace{0.3em}
Echtzeit-OS f�r das Internet der Dinge 			\\							% Untertitel eingeben
\vspace{5mm}

\fontsize{10pt}{12pt}\selectfont
\textbf{Projektarbeit} \\									% eingeben
\vspace{3mm}

% Abstract (eingeben):
%---------------------------------------------------------------------------
\begin{textblock}{150}(28,190)
\fontsize{10pt}{12pt}\selectfont
Die Linux Foundation hat mit dem Projekt Zephyr mit der Entwicklung eines Echtzeit-Betriebssystems f�r das Internet der Dinge (IoT) begonnen.
Zephyr ist ein Open-Source-Betriebssystem mit dem Ziel ein solides OS f�r IoT Ger�te mit geringen Ressourcen bereitzustellen. Es nutzt eine echtzeitf�hige Kombination aus Nano- und Microkernel. 
Im Gegensatz zu einem Linux Kernel ben�tigt Zephyr nur zwischen 8 und 512 KByte an Arbeitsspeicher.  Aktuell werden folgenden Plattformen unterst�tzt: x86, ARM und ARC EM4
\end{textblock}

\begin{textblock}{150}(28,225)
\fontsize{10pt}{17pt}\selectfont
\begin{tabbing}
xxxxxxxxxxxxxxx\=xxxxxxxxxxxxxxxxxxxxxxxxxxxxxxxxxxxxxxxxxxxxxxx \kill
Studiengang:	\> Elektro- und Kommunikationstechnik			\\
Autoren:		\> Aaron Schmocker, David Wyss					\\
Betreuer:		\> Martin Aebersold								\\
Auftraggeber:	\> Martin Aebersold								\\
Experten:		\> Martin Aebersold								\\
Datum:			\> \versiondate									\\
\end{tabbing}

\end{textblock}
\end{flushleft}

\begin{textblock}{150}(28,280)
\noindent 
\color{bfhgrey}\fontsize{9pt}{10pt}\selectfont
Berner Fachhochschule | Haute �cole sp�cialis�e bernoise | Bern University of Applied Sciences
\color{black}\selectfont
\end{textblock}


\end{titlepage}

%
% ===========================================================================
% EOF
%
                      % activate for Titelseite mit Bild
% Versionenkontrolle :
% -----------------------------------------------

\color{black}
%\begin{huge}
\chapter*{Versionen}
%\end{huge}
\vspace{10mm}

\fontsize{10pt}{18pt}\selectfont
\begin{tabbing}
xxxxxxxxxxx\=xxxxxxxxxxxxxxx\=xxxxxxxxxxxxxx\=xxxxxxxxxxxxxxxxxxxxxxxxxxxxxxxxxxxxxxxxxxxxxxx \kill
Version	\> Datum	\> Status		\> Bemerkungen		\\
0.1	\> 15.09.2016	\> Entwurf		\> Erster Entwurf	\\	
	
\end{tabbing}



\cleardoubleemptypage
\setcounter{page}{1}
\cleardoublepage
\phantomsection 
\addcontentsline{toc}{chapter}{Management Summary}
% !TeX encoding = ISO-8859-1
\chapter*{Management Summary}
\label{chap:managementSummary}

% Abstract (eingeben):
%---------------------------------------------------------------------------

Diese Arbeit bietet einen �berblick �ber das Betriebssystem Zephyr mit Schwerpunkt auf Technologien, Protokolle und Anwendungsfragen. Das Zehphyr Beriebssystem ist speziell f�r das Internet der Dinge(IoT), welches durch die neuesten Entwicklungen in den Bereichen RFID, intelligente Sensoren, Kommunikationstechnologien und Internet-Protokolle schnell w�chst, entwickelt worden. Die Grundvoraussetzung ist, dass intelligente Sensoren direkt ohne menschliches Engagement zusammenarbeiten, um eine neue Klasse von Anwendungen zu liefern. Die aktuelle Revolution in den Bereichen Internet, Mobile und Machine-to-Machine (M2M) ist die erste Phase des IoT. In den kommenden Jahren wird erwartet, dass die IoT diverse Technologien �berbr�ckt, um neue Anwendungen durch die Verbindung von physischen Objekten zur Unterst�tzung intelligenter Entscheidungsfindung zu erm�glichen.  Die Linux Foundation hat mit dem Projekt Zephyr mit der Entwicklung eines Echtzeit-Betriebssystems f�r das Internet der Dinge (IoT) begonnen. Zephyr ist ein Open-Source-Betriebssystem mit dem Ziel ein solides OS f�r IoT Ger�te mit geringen Ressourcen bereitzustellen. Es nutzt eine echtzeitf�hige Kombination aus Nano- und Microkernel. Im Gegensatz zu einem Linux Kernel ben�tigt Zephyr nur zwischen 8 und 512 KByte an Arbeitsspeicher.  Aktuell werden folgenden Plattformen unterst�tzt: x86, ARM und ARC.

Dieses Arbeit beginnt mit einem horizontalen �berblick �ber die Zehphyr Beriebssystem. Anschliessend wird detailiert der Aufbau und die Ziele des Betriebsysstems erl�utert und einen �berblick �ber einige technische Details,wie z.B. Technologien, Protokolle und Anwendungen erm�glicht. Im Vergleich zu anderen Berichten, sofern diese schon vorliegen in diesem Bereich, ist es unser Ziel, eine umfassendere Zusammenfassung der wichtigsten Protokolle und Anwendungsprobleme zu bieten, damit Forscher und Anwendungsentwickler einen schnellen �berblick �ber die gew�nschten Funktionalit�ten erhalten. Des Weiteren bieten wir auch einen �berblick �ber einige andere RTOS an, um die Vor- und Nachteile des Zephyr Betriebssystems besser auf zu zeigen. Ausserdem schrieben wir eine Anleitung wie genau mau nund unsere verwendetes Noridc Eval-Board in betrieb nehmen kann und wie unsere Applikation aufgebauen ist und wie man sie bedient. Das Ziel der Arbeit
war, mit dem neuen Betreibsystem Zephyr Erfahrungen zusammeln. Evaluieren, welches Betriebssystem sich am besten f�r das Internet der Dinge (IoT) und die passenden Sensorknoten eignet, um diese Erkenntnisse in sp�teren Projekten der Berner Fachhochschule einsetzen zu k�nnen.

 


\cleardoubleemptypage
%---------------------------------------------------------------------------

% Table of contents
%---------------------------------------------------------------------------
\tableofcontents
\cleardoublepage
%---------------------------------------------------------------------------

% Main part:
%---------------------------------------------------------------------------
\pagenumbering{arabic}

% !TeX encoding = ISO-8859-1
\chapter{Einleitung}
\label{chap:einleitung}

\section{TEST}


\newcommand*{\Package}[1]{\texttt{#1}}
\noindent The command:
\lstinline[style=BashInputStyle]�# apt-get --purge remove rubygems�.
removes the \Package{rubygems} package.

\bigskip
\noindent Consider the following command:
\begin{lstlisting}[style=BashInputStyle]
# apt-get --purge remove rubygems
\end{lstlisting}
This removes the \Package{rubygems} package.


\section{TEST2}



% !TeX encoding = ISO-8859-1
\chapter{Zephyr-Overview}
\label{chap:overview}

\section{�bersicht Zephyr-OS}

Zephyr ist laut Beschreibung der Linux-Foundation \cite{LinuxFoundation} ein Open-Source-Echtzeitbetriebssystem, speziell optimiert f�r Anwendungen im Internet der Dinge. Konkret also f�r Gl�hlampen, Heizungsthermostate und andere Alltagsger�te mit Internet-Anbindung.

Die Architektur basiert nicht wie bei anderen IoT-Betreibssystemen auf einem Linux-Kernel, sondern auf einer echtzeitf�higen Kombination von Nano- und Mikrokernel. Zephyr wird aktuell von der Linux-Foundation in Zusammenarbeit mit den Firmen Intel, NXP und Synopsys in der Form eines Collaborative-Projects weiterentwickelt. Dadurch soll versucht werden die bei der Linux- und Open-Source-Entwicklung verwendeten Arbeitsweisen und Ideen auch im Bereich der Industrie einzubringen.

Ziel ist es ein robustes und sicheres Betriebssystem f�r das Internet der Dinge zu schaffen. Zephyr ist dabei vollst�ndig Open-Source und steht laut Information des Newsportals Heise.de \cite{HeiseDe} unter der Apache Lizenz Version 2.0. 
Dieses Lizenzierungsmodell kommt Firmen und Unternehmen entgegen, welche den Einsatz von Open-Source-Software generell scheuen. Open-Source Projekte stehen oft unter der \ac{GPL}. [2] Wird in Produkten Software verwendet welche unter der \ac{GPL} lizensiert ist, zwingt dieses Lizenzmodell die Firmen dazu ihre Produkte ebenfalls unter der \ac{GPL} zu ver�ffentlichen. Dies beinhaltet auch s�mtliche �nderungen welche vorgenommen wurden. Bei der Apache-Lizenz ist dies gem�ss Definition \cite{ApacheLicense} nicht zwingend.

\begin{figure}[h]
	\centering
	\includegraphics[width=0.7\linewidth]{bilder/zephyr_components.jpg}
	\caption{Komponente und �bersicht �ber das Zephyr RTOS}
	\label{fig:components}
\end{figure}

\newpage

\section{Ziele}

Zephyr ist f�r den Einsatz auf Ger�ten mit geringem Speicherplatz und feststehender Hardwarekonfiguration gedacht. Darunter fallen unter anderem Steuerungen f�r Heizungs- und Beleuchtungssysteme und auch Ger�te aus allen Bereichen des t�glichen Lebens mit Internet-Anbindung.

Das ZephyrProjekt verfolgt laut Beschreibung der Linux-Foundation \cite{LinuxFoundation} folgende Ziele:

\begin{itemize}
	\item Kleine Speicheranforderung - lauff�hig mit minimal 10kB
	\item CPU unabh�ngige Architektur mit abstrahierten Funktionen
	\item Modular und hochgradig skalierbar
	\item Hohe Sicherheitsstandards
	\item Unterst�tzt auf Kernellevel viele unterschiedliche Boards und Kommunikationsprotokolle
	\item M�chtige Entwicklungswerkzeuge
	\item OpenSource-Kernel mit Apache v2.0 Lizenz
\end{itemize}

\section{Aufbau}

Das Zephyr OS setzt auf eine Kombination von Nano- und Mikrokernel. Dadurch soll Zephyr bereits mit 10KB an Speicherplatz lauff�hig sein. Das macht Zephyr besonders f�r Anwendungen auf kleinen Mikrocontrollern attraktiv. Bei einem herk�mmlichen Linux-Kernel w�re dies nicht denkbar. G�ngige Adaptionen f�r Smartphone SoCs ben�tigen laut [1] in der kleinsten Konfiguration noch bis zu 200KB RAM und rund 1MB Flash.

\begin{figure}[h]
	\centering
	\includegraphics[width=0.6\linewidth]{bilder/zephyr_rtos_concept.jpg}
	\caption{�bersicht �ber den Nano- und Mikrokernel des Zephyr-Betriebssytems}
	\label{fig:components}
\end{figure}

Der Nanokernel bietet Echtzeit-F�higkeiten. Die Zeit die der Nanokernel f�r die Abarbeitung einer Aufgabe ben�tigt verh�lt somit deterministisch. Das bedeutet dass genau festgelegt werden kann wieviel Zeit der Kernel f�r eine zuvor festgelegte Aufgabe ben�tigen wird. Dies unabh�ngig davon wie stark das System gerade ausgelastet ist. F�r alle Aufgaben welche keine Anforderungen an Echtzeit-Verarbeitung stellen verf�gt das System �ber einen Microkernel. 

\section{Unterst�tztung}

Momentan unterst�tzt der Zephyr-Kernel gem�ss Angaben auf der Projektseite \cite{ZephyrProjectDocumentation} Prozessoren der Architekturen ARC, ARM-v7 sowie auch x86. Dadurch ist das System auf den zur Zeit popul�ren Plattformen lauff�hig.
Zephyrapplikationen k�nnen mittels SDK aber auch f�r System-Emulatoren wie QEMU kompiliert werden. Folgende Hardware wird zum Zeitpunkt unserer Projektarbeit unterst�tzt:

\begin{center}
	\begin{tabular}{ | l | l | l |}
		\hline
		\textbf{ARC}		& \textbf{ARM} 		& \textbf{X86}  \\ \hline
		Arduino 101			& 96B-Carbon		& Galileo Gen1 Gen2   \\ \hline
		EDesignWare EM		& 96B-Nitrogen		& Minnowboard Max \\ \hline
		Emulation / QEMU	& Arduino Due		& Quark D2000 CRB	\\ \hline
							& Hexiwear		 	& Emulation / QEMU	\\ \hline
							& NXP FRDM-K64		&					\\ \hline
							& OLIMEXINO-STM32	&					\\ \hline	
							& nRF51-PCA10028	&					\\ \hline
							& nRF52-PCA10040	&					\\ \hline
							& nRF52840-PCA10056	&					\\ \hline
							& V2M Beetle		&					\\ \hline
							& Emulation / QEMU	&					\\ 
		\hline
	\end{tabular}
\end{center}

Die Zephyr Codebasis unterst�tzt zum Zeitpunkt der Projektarbeit von Haus aus eine Vielzahl an Kommunikationsprotokollen. Da der Fokus auf das Internet der Dinge gelegt wurde, wurden auch die entsprechenden Protokolle zuerst implementiert. Zweifelsohne werden im Verlaufe der Entwicklung noch weitere folgen.

Folgende Tabelle zeigt die Kommunikationsm�glichkeiten sowohl die vom Betriebssystem unterst�tzte Hardwarekomponenten wie auch die unterst�tzten Protokolle.

\begin{center}
	\begin{tabular}{ | l | l |}
		\hline
		\textbf{Protokoll} 		& \textbf{Hardwarekomponente} \\ \hline
		Bluetooth 4.0 			& ADC 	\\ \hline
		Bluetooth Low Energy 	& GPIOs \\ \hline
		CoAP 					& I2C 	\\ \hline
		HTTP					& IPM 	\\ \hline
		IEEE 802.15.4			& PWM	\\ \hline
		IPv4 / IPv6 			& SPI	\\ \hline	
		MQTT		 			& UART	\\ \hline
		6LoWPAN 				& 		\\ \hline
		Radios 					& 		\\ \hline
		Wi-Fi					& 		\\
		\hline
	\end{tabular}
\end{center}

\newpage

\section{Aufsetzen der Entwicklungsumgebung}

Der folgende Abschnitt beschreibt das Aufsetzen der Zephyr-SDK unter Linuxsystemen. Die Anleitung funktioniert f�r beliebige Betriebssysteme, wenn die Paketquellen entsprechend angepasst werden.

\subsection{Notwendige Software und SDK}

Folgender Befehl installiert alle f�r die Zephyr-SDK notwendigen Pakete.

\begin{lstlisting}[style=BashInputStyle]
$ apt-get install git make gcc g++ python3-ply ncurses-dev
\end{lstlisting}

Die SDK von Zephyr verf�gt �ber alle n�tigen Tools und Crosscompiler um den Kernel f�r alle supporteten Architekturen zu kompilieren. Zus�tzlich enth�lt sie eine speziell angepasste Version von QEMU mit welcher sich die Architekturen auf einem beliebigen Hostsystem simulieren lassen.

Die Installation der SDK gliedert sich in folgende Schritte:

1) Herunterladen der neuesten SDK Version:

\begin{lstlisting}[style=BashInputStyle]
$ wget https://nexus.zephyrproject.org/content/repositories/${VERSION}
\end{lstlisting}

2) Ausf�hren der heruntergeladenen Datei:

\begin{lstlisting}[style=BashInputStyle]
$ chmod +x zephyr-sdk-<version>-i686-setup.run
$ ./zephyr-sdk-<version>-i686-setup.run
\end{lstlisting}

3) Folgen der ausgegebenen Instruktionen auf dem Bildschirm. Die SDK wird per Standard im Ordner /opt/zephyr-sdk/ abgelegt. Auf den meisten Systemen wird daf�r ein Administratorzugang vorausgesetzt, da es sich bei /opt um ein Systemverzeichnis handelt. Optional kann die SDK auch im /home des Users installiert werden.

4) Um die Zephyr-SDK verwenden zu k�nnen m�ssen folgende Umgebungsvariablen gesetzt werden: 

\begin{lstlisting}[style=BashInputStyle]
$ export ZEPHYR_GCC_VARIANT=zephyr
$ export ZEPHYR_SDK_INSTALL_DIR=${Installationspfad}
\end{lstlisting}

Es gilt zu beachten, dass diese Umgebungsvariablen immer nur f�r die aktuelle Shell gelten. Sollen diese automatisch mit der Shell geladen werden k�nnen die Instruktionen zur Shellkonfiguration (bashrc) hinzugef�gt werden. 

Eine weitere m�glichkeit ist die zephyrrc welche im Home-Verzeichnis angelegt werden kann.

\begin{lstlisting}[style=BashInputStyle]
$ cat <<EOF > ~/.zephyrrc
export ZEPHYR_GCC_VARIANT=zephyr
export ZEPHYR_SDK_INSTALL_DIR=/opt/zephyr-sdk
EOF
\end{lstlisting}

\subsection{Herunterladen des Zephyr-Softwarearchives}

Das gesamte Repository von Zephyr wird von der Linux Foundation mit einem Gerrit-Backend zur Verf�gung gestellt. Der Code l�sst sich bequem mittels git klonen.

Folgender Befehl l�dt das gesamte Zephyr-Repository in den Ordner zephyr-project herunter:

\begin{lstlisting}[style=BashInputStyle]
$ git clone https://gerrit.zephyrproject.org/r/zephyr zephyr-project
\end{lstlisting}

Um an der Entwicklungsarbeit teilzunehmen ist ein Linux Foundation Account notwendig. Dieser l�sst sich im Gerrit Account-Bereich der Zephyr-Webseite mit wenigen Angaben erstellen.

\subsection{Testen der Toolchain mittels QEMU}

Die Beispielapplikationen befinden sich im Zephyr-Projektordner unter der Kategorie Samples. Sie lassen sich in nur wenigen Schritten f�r QEMU kompilieren und ausf�hren.

1) Sicherstellen dass die SDK korrekt installiert wurde und die n�tigen Umgebungsvariablen gesetzt sind. 

2) Navigieren in den Projektordner.

3) Bereitstellen aller Umgebungsvariablen durch einlesen des zur Verf�gung gestellten Skripts:

\begin{lstlisting}[style=BashInputStyle]
$ source zephyr-env.sh
\end{lstlisting}

4) Kompilieren des gew�nschten Beispielprojekts.

\begin{lstlisting}[style=BashInputStyle]
$ cd $ZEPHYR_BASE/samples/hello_world
$ make
\end{lstlisting}

Obige Instruktionen bauen das Hello World Beispiel aus dem Quellcode. Als Standard ist im Makefile dieses Beispiels QEMU als Zielarchitektur vorgesehen. Sollte eine andere Architektur gew�nscht sein kann dies dem Makefile wie folgt mitgeteilt werden

\begin{lstlisting}[style=BashInputStyle]
$ make BOARD=nrf52
\end{lstlisting}

Falls eine Hilfestellung gew�nscht ist kann diese mit foilgendem Kommando aufgerufen werden.

\begin{lstlisting}[style=BashInputStyle]
$ make help
\end{lstlisting}

Die erstelten ELF-Binaries und HEX-Files befinden sich jeweils im Projektordner im Ordner outdir. Wurden diese Dateien f�r spezifische Boards kompiliert k�nnen diese gleich mit folgendem Befehl auf das Board geladen werden:

\begin{lstlisting}[style=BashInputStyle]
$ make flash
\end{lstlisting}

Es gilt zu beachten dass die Flash-Instruktion momentan nur wenige Boards unterst�tzt. Das nrf52dk, auf welches sp�ter genauer eingegangen wird ist nicht unterst�tzt und muss eigens mit einem Programm von Nordic angesprochen werden.

\subsection{Simulation mit QEMU f�r verschiedene Architekturen}

Der Emulator QEMU ist f�hig Software f�r alle von Zephyr unterst�tzten Architekturen auszuf�hren. Dazu muss der Programmcode mit den richtigen Argumenten kompiliert werden. Folgendes Beispiel zeigt das Kompilieren des Codes f�r X86er Architekturen in QEMU. Durch das Argument qemu am Schluss des Aufrufes von Make wird nach erfolgreichem Kompilieren automatisch der Emulator gestartet. 

\begin{lstlisting}[style=BashInputStyle]
$ make BOARD=qemu_x86 qemu
\end{lstlisting}
 
 




 % !TeX encoding = ISO-8859-1
 \chapter{FreeRTOS-Overview}
 \label{chap:overview}
 
 \section{�bersicht �ber FreeRTOS}
 
 FreeRTOS ist laut Beschreibung der Real Time Enginners Ltd \cite{LinuxFoundation} ein Open-Source-Echtzeitbetriebssystem. FreeRTOS wurde professionell entwickelt und wird gewartet durch die Firma Real Time Enginners Ltd. Sie arbeiten seit �ber 12 Jahren in enger Partnerschaft mit weltweit f�hrenden Chip-Herstellern zusammmen, um ihren Kunden eine v�llig freie Qualit�tssoftware zur Verf�gung zu stellen. Mittels hohem C Quellcodes Standart wurde das robuste Echtzeitbetriebssystem zum Marktf�hrer unter den RTOS und unterst�tz �ber 35 Architekturen.
 
 FreeRTOS ist sehr streng verwaltet, nicht nur in Software-Coding-Standards und Look-and-Feel, sondern auch in der Umsetzung. Beispielsweise:
 
  \begin{itemize}
  	\item  FreeRTOS f�hrt nie einen nicht-deterministischen Vorgang aus, wie etwa das Verfolgen einer verkn�pften Liste, und zwar innerhalb eines kritischen Abschnitts oder Interrupts.
  	\item Wir sind besonders stolz auf die effiziente Software-Timerimplementierung, die keine CPU-Zeit in Anspruch nimmt, es sei denn, ein Timer muss tats�chlich gewartet werden. Software-Timer enthalten keine Variablen, die bis auf Null gez�hlt werden m�ssen.
  	\item Ebenso ben�tigen Listen von blockierten (angeh�ngten) Tasks keine zeitraubende periodische Wartung.
  	\item Direkt zu Task-Benachrichtigungen erm�glichen eine schnelle Task-Signalisierung mit praktisch keinem RAM-Overhead und k�nnen in der Mehrzahl der Inter-Task verwendet werden.
  	\item Das FreeRTOS-Warteschlangen-Nutzungsmodell schafft es, Einfachheit mit Flexibilit�t (in einer winzigen Codegr��e) zu kombinieren - Attribute, die sich normalerweise gegenseitig ausschlie�en.
  \end{itemize}
 
  
  \begin{figure}[h]
  	\centering
  	\includegraphics[width=0.7\linewidth]{bilder/FreeRTOS_Context.jpg}
  	\caption{�bersicht �ber das FreeRTOS}
  	\label{fig:components}
  \end{figure}
  
 
Es wird unter der GPL mit einer zus�tzlichen Einschr�nkung und optionaler Ausnahme verteilt. Die Einschr�nkung verbietet das Benchmarking, w�hrend die Ausnahme erlaubt, dass der propriet�re Code der Benutzer eine geschlossene Quelle bleibt, w�hrend der Kernel selbst als Open Source beibehalten wird, wodurch die Verwendung von FreeRTOS in propriet�ren Anwendungen erleichtert wird. FreeRTOS wurde schon im Weltraum eingesetzt.\cite{FreeRTOSProjectDocumentation}
 
 \section{Ziele}
 
 Die urspr�ngliche Aufgabe des FreeRTOS-Projekts war es, eine kostenlose RTOS-L�sung bereitzustellen, die einfach zu bedienen war. Das heisst, einfach zu erstellen und zu implementieren, auf einem Windows- oder Linux Computer, ohne herauszufinden, welche Quelldateien und welche Pfade erforderlich sind, oder wie die Echtzeit-Debugging-Umgebung konfiguriert wird. Dies wurde durch die Bereitstellung von vorkonfigurierten Beispielprojekten f�r jeden offiziell unterst�tzten Board erreicht.
 
 FreeRTOS ist ein skalierbarer Echtzeitkern, der speziell f�r kleine Embedded-Systeme entwickelt wurde. Zu den Highlights geh�ren:\cite{FreeRTOSProjectDocumentation}
 
 
 \begin{itemize}
 	\item Freier RTOS-Scheduler - pr�ventive, kooperative und hybride Konfigurationsoptionen mit optionalem Time Slicing
 	\item CPU unabh�ngige Architektur
 	\item SafeRTOS-Produkt soll ein hohes Ma� an Vertrauen in die Codeintegrit�t schaffen
 	\item Enth�lt einen Tickless-Modus f�r Anwendungen mit geringer Leistung
 	\item Unterst�tzt viele unterschiedliche Boards und Kommunikationsprotokolle
 	\item RTOS-Objekte (Tasks, Warteschlangen, Semaphoren, Software-Timer, Mutexes und Ereignisgruppen) k�nnen entweder mit dynamisch oder statisch zugewiesenem RAM erstellt werden
 	\item Unterst�tzt sowohl Echtzeitaufgaben als auch Co-Routinen.
 	\item Mutexes mit Priorit�tsvererbung
 	\item Effiziente Software-Timer
 	\item Leistungsstarke Ablaufverfolgungsfunktionalit�t
	\item FreeRTOS-MPU unterst�tzt die ARM Cortex-M3 Memory Protection Unit (MPU)
 \end{itemize}
 

 \section{Aufbau}
 
 Die Implementierung FreeRTOS ist so konzipiert, dass es klein und einfach ist. Der Kernel selbst besteht aus nur drei C-Dateien. Um den Code lesbar, einfach zu portieren und wartbar zu machen, ist er meistens in C geschrieben, aber es gibt, wenn n�tig, einige Montagefunktionen (meistens in architekturspezifischen Schedulerroutinen). FreeRTOS bietet Methoden f�r mehrere Threads oder Aufgaben, Mutexes, Semaphoren und Software-Timer. Ein Tick-less-Modus ist f�r Anwendungen mit geringer Leistung vorgesehen. Thread-Priorit�ten werden unterst�tzt. Dar�ber hinaus gibt es vier Schemata der Speicherzuweisung:

   \begin{itemize}
   	\item Nur Zuweisung
   	\item Zuweisen und kostenlos mit einem sehr einfachen, schnellen Algorithmus
   	\item Ein komplexer, aber schnell zuweisbarer und freier Algorithmus mit Speicherkoaleszenz 
   	\item C-Bibliothek zuzuordnen und kostenlos mit einigen gegenseitigen Ausschluss-Schutz
   \end{itemize}
   \vspace{5mm}
   

 In FreeRTOS gibt es keine Erweiterungsfunktionen wie z.B: Ger�tetreiber, erweiterte Speicherverwaltung und Benutzerkonten wie das bei Betriebssystemen wie Linux oder Microsoft Windows �blich ist. Der Schwerpunkt liegt auf Kompaktheit und Geschwindigkeit der Ausf�hrung:
 
   \begin{itemize}
   	\item Sehr geringer Speicherbedarf, geringer Overhead und sehr schnelle Ausf�hrung
   	\item Tick-less Option f�r Low-Power-Anwendungen
   	\item Gleicherma�en gut f�r Hobbyisten, die neu f�r Betriebssysteme sind und professionelle Entwickler, die an kommerziellen Produkten arbeiten
   	\item   Der Scheduler kann sowohl f�r den pr�ventiven als auch f�r den kooperativen Betrieb konfiguriert werden
   	\item Coroutine Unterst�tzung (Coroutine in FreeRTOS ist eine sehr einfache und leichte Aufgabe, die sehr begrenzte Nutzung von Stack hat)
   \end{itemize}
   \vspace{5mm}
 
 FreeRTOS kann als 'Thread-Bibliothek' und nicht als 'Betriebssystem' gedacht werden, obwohl Kommandozeilen-Interface und POSIX-�hnliche I / O Abstraktions-Add-ons zur Verf�gung stehen. FreeRTOS implementiert mehrere Threads, indem das Host-Programm eine Thread-Tick-Methode in regelm��igen kurzen Abst�nden aufrufen. Die Thread-Tick-Methode wechselt Tasks abh�ngig von Priorit�t und einem Round-Robin-Scheduling-Schema. Das �bliche Intervall betr�gt 1/1000 einer Sekunde bis 1/100 Sekunde �ber eine Unterbrechung von einem Hardware-Zeitgeber, aber dieses Intervall wird h�ufig ge�ndert, um es einer bestimmten Anwendung anzupassen.
 
 Der Download enth�lt vorbereitete Konfigurationen und Demonstrationen f�r jeden Board und Compiler, was ein schnelles Anwendungsdesign erm�glicht.\cite{FreeRTOSProjectDocumentation} 
 
 
 \section{Unterst�tzung}
 
 Momentan unterst�tzt der FreeRTOS-Kernel, gem�ss Angaben auf der Projektseite \cite{FreeRTOSProjectDocumentation}, Prozessoren der Architekturen ARM. Somit  kann es sowohl auf einem System-Emulator wie z.B. Qemu kompiliert werden oder auf einer unterst�tzten Hardware. Folgende Hardware-Hersteller wurde zum Zeitpunkt unserer Projektarbeit unterst�tz:
 
 \begin{center}
 \begin{tabular}{ | l |}
 	\hline
 	\textbf{Hersteller} 	\\ \hline
 	Altera Nios II 			\\ \hline
 	ARM architecture 		\\ \hline
 	Atmel 					\\ \hline
 	Cortus 					\\ \hline
 	Cypress					\\ \hline
 	Energy Micro			\\ \hline
 	Fujitsu					\\ \hline
 	Freescale				\\ \hline
 	IBM						\\ \hline	
	Infineon				\\ \hline	
 	Intel 					\\ \hline
 	NXP						\\ \hline
 	PIC microcontroller		\\ \hline
 	STMicroelectronics		\\ \hline
 	Texas Instruments		\\ \hline
 	Xilinx					\\ \hline
 	\hline
 \end{tabular}
\end{center}
 
 \vspace{4mm}
 
 Zur Kommunikation unterst�tzt das FreeRTOS-Betriebssystem eine Vielzahl an Kommunikations-Protokolle. Da der Fokus auf das Internet der Dinge gelegt wurde, wurden auch die entsprechenden Protokolle zuerst implementiert. 
 
 Folgende Tabelle zeigt die Kommunikationsm�glichkeiten sowohl die vom Betriebssystem unterst�tzte Hardwarekomponenten wie auch die unterst�tzten Protokolle:
 
 
 \begin{center}
 	\begin{tabular}{ | l | l |}
 		\hline
 		\textbf{Protokoll} 		& \textbf{Hardwarekomponente} \\ \hline
 		Bluetooth 4.0 			& CAN 	\\ \hline
		IPv4 / IPv6				& GPIO 	\\ \hline
 		TCP			 			& I2C	\\ \hline
 		HTTP 					& SPI	\\ \hline	
 		6LoWPAN 				& UART	\\ \hline
 		Wi-Fi					& 		\\ \hline
 		\hline
 	\end{tabular}
 \end{center}
 


 

 % !TeX encoding = ISO-8859-1
 \chapter{RIOT-Overview}
 \label{chap:overview}
 
 \section{�bersicht �ber RIOT}
 
 RIOT ist ein Open Source-Mikrokernel-basiertes Betriebssystem, das auf die Anforderungen von IOT-Ger�ten und anderen eingebetteten Ger�ten abgestimmt ist. Diese Anforderungen umfassen einen sehr geringen Speicherbedarf in der Gr��enordnung von einigen Kilobytes, eine hohe Energieeffizienz, Echtzeitf�higkeiten, Kommunikationsstapel f�r sowohl drahtlose als auch drahtgebundene Netzwerke und Unterst�tzung f�r eine breite Palette von Niedrigleistungs-Hardware.
 RIOT stellt einen Mikrokernel, mehrere Netzwerkstapel und Dienstprogramme bereit, die kryptografische Bibliotheken, Datenstrukturen, Hash-Tabellen und eine Shell umfasst. RIOT unterst�tzt eine breite Palette von Mikrocontroller-Architekturen, Radiotreibern, Sensoren und Konfigurationen f�r ganze Plattformen. �ber alle unterst�tzten Hardware 32-Bit-, 16-Bit- und 8-Bit-Plattformen bietet RIOT eine konsistente API und erm�glicht ANSI C und C ++ Anwendungsprogrammierung, mit Multithreading, IPC, System-Timer und Mutexes. Es ist unter der LGPL lizenziert und wird von der Freien Universit�t Berlin, dem Institut national de recherche en informatique et en automatique (INRIA) und der Hochschule f�r Angewandte Wissenschaften Hamburg entwickelt.\cite{RIOTProjectDocumentation}
 

 \section{Ziele}
 Das Vorg�ngerprojekt von RIOT hie� Feuerware und war als Betriebssystem f�r drahtlose Sensornetzwerke gedacht. Entwickelt wurde es im Rahmen des FeuerWhere Projekts, das Feuerwehrm�nner im Einsatz �berwachen sollte. 2008 wurde an der Freien Universit�t Berlin mit der Entwicklung begonnen. Im Jahr 2010 kam es zu einer Abspaltung (Fork) von Feuerware und das Programm wurde in RIOT umbenannt. Damit einhergehend wurde IETF-Protokolle wie etwa 6LoWPAN, RPL und TCP implementiert, um es f�r einen Einsatz im Internet anzupassen. 2013 erfolgte die Umbenennung in RIOT. 
  
  \begin{figure}[h]
  	\centering
  	\includegraphics[width=1.0\linewidth]{bilder/RIOT.png}
  	\caption{�bersicht �ber das RIOT}
  	\label{fig:components}
  \end{figure} 
  \vspace{5mm} 
   
 RIOT-Programme k�nnen in C und C++ geschrieben werden. Es ist im Gegensatz zu anderen kleinen Betriebssystemen wie TinyOS echtes Multithreading verf�gbar. F�r Linux und MacOS existieren native Portierungen, so dass Anwendungen auf dem Computer geschrieben und dann schnell auf echte Hardware portiert werden k�nnen, was das Debugging erleichtern soll. Dabei werden Standardwerkzeuge wie GNU Compiler Collection (GCC), GNU Debugger benutzt.[3] Aufgrund der Herkunft als Betriebssystem f�r Sensornetzwerke bei der Feuerwehr ist RIOT echtzeitf�hig. Teile des POSIX-Standards sind implementiert.
 Der Quellcode liegt auf GitHub und wird von einer freien Entwickler-Community mitentwickelt.\cite{RIOTProjectDocumentation}
 
 \section{Aufbau}
 
 Dieser Abschnitt f�hrt Sie durch die Struktur von RIOT. Sobald Sie diese Struktur verstehen, verstehen Sie den RIOT-Code-Basis.
 
 \vspace{5mm} 
 \begin{figure}[h]
  \centering
  \includegraphics[width=0.8\linewidth]{bilder/RIOT.jpg}
  \caption{Struktur von RIOT}
  \label{fig:components}
 \end{figure}
 \vspace{10mm}

 
 Die Codebasis von RIOT ist in vier Gruppen eingeteilt.
  
 \begin{itemize}
   \item Der Kernel (Core)
   \item Plattformspezifischer Code (CPU)
   \item Ger�tetreiber (Drivers)
   \item Bibliotheken und Netzwerk-Code (sys; pkg)
 \end{itemize}
 \vspace{4mm}
  
  Dar�ber hinaus enth�lt RIOT eine Sammlung von Skripten f�r verschiedene Aufgaben, sowie eine vordefinierte Umgebung f�r die Erstellung dieser Dokumentation. Die Strukturgruppen werden auf die Verzeichnisstruktur von RIOT projiziert, wobei jede dieser Gruppen in einem oder zwei Verzeichnissen im Haupt-RIOT-Verzeichnis liegt. Die folgende Liste enth�lt eine detaillierte Beschreibung der einzelnen Verzeichnisse des RIOTs:
  
  
  \vspace{5mm}
  
  Core:
  
  Dieses Verzeichnis enth�lt den eigentlichen Kernel. Der Kernel besteht aus dem Scheduler, der Interprozess-Kommunikation (Messaging), dem Threading, der Threadsynchronisation und den unterst�tzenden Datenstrukturen und Typdefinitionen.
  
  \vspace{5mm}
  
  Board:
  
  Der plattformabh�ngige Code ist in zwei logische Elemente CPU und Board unterteilt. Eine Board hat genau eine CPU, w�hrend eine CPU Teil von vielen Boards sein kann. Der CPU-Teil enth�lt alle generischen, CPU-spezifischen Code.
  
  Der Board-Teil enth�lt die spezifische Konfiguration f�r die darin enthaltene CPU. Diese Konfiguration umfasst haupts�chlich die Peripheriekonfiguration und Pin-Mapping, die Konfiguration von On-Board-Ger�ten und die Taktkonfiguration der CPU. Zus�tzlich zu den Quell- und Header-Dateien, die f�r jedes Board ben�tigt werden, kann dieses Verzeichnis zus�tzlich einige Skript- und Konfigurationsdateien enthalten, die f�r die Anbindung an die Boards ben�tigt werden. 
  
  \vspace{5mm}
  
  CPU:
  
  F�r jede unterst�tzte CPU enth�lt dieses Verzeichnis ein Unterverzeichnis mit dem Namen der CPU. Diese Verzeichnisse enthalten dann alle CPU-spezifischen Konfigurationen, wie Implementierungen des Energiemanagements (LPM), Interrupt-Handler und Vektoren, Startupcode und Taktinitialisierungscode. F�r die meisten CPUs finden Sie auch die Linkerskripte im Unterverzeichnis.
  
  Im Periph-Unterverzeichnis jeder CPU finden Sie die Implementierungen der Peripherie-Treiber wie SPI, UART, GPIO, etc. Viele CPUs teilen einen bestimmten Code, z.B. alle ARM Cortex-M-basierten CPUs verwenden denselben Code f�r den Taskwechsel und den Interrupthandler. Dieser freigegebene Code hat ein eigenes Verzeichnis.
  
  \vspace{5mm}
  
  Drivers:
  
  Dieses Verzeichnis enth�lt die Treiber f�r externe Ger�te wie Netzwerkschnittstellen, Sensoren und Aktoren. Jeder Ger�tetreiber wird in ein eigenes Unterverzeichnis mit dem Namen des Ger�ts eingef�gt.
  
  Alle Ger�tetreiber des RIOT basieren auf der peripheren Treiber-API z.B. SPI, GPIO und anderen RIOT-Modulen wie dem xtimer. Auf diese Weise sind die Treiber v�llig plattformunabh�ngig.
  
  \vspace{5mm}
  
  Sys /net:
  
  RIOT folgt dem Mikrokern-Design-Paradigma, wo alles ein Modul sein soll. Alle diese Module, die nicht Teil der Hardware-Abstraktion oder Ger�tetreiber sind, k�nnen in diesem Verzeichnis gefunden werden. Die Bibliotheken umfassen Datenstrukturen, Kryptobibliotheken, Hight-level APIs und Speicherverwaltung. 

  Das Unterverzeichnis sys / net muss explizit erw�hnt werden, da hier der gesamte Netzwerkcode im RIOT liegt. Hier finden Sie die Netzwerk-Stack-Implementierungen z.B. den GNRC-Stack.
  
  \vspace{5mm}
  
  Pkg:
  
  RIOT unterst�tzt mehrere externe Bibliotheken z.B. OpenWSN, Microcoap. Die exteren Bibliotheken werden als benutzerdefinierte Makefiles ausgeliefert. Die Bibliothek l�dt optional eine Anzahl von Patches herunter, damit RIOT funktioniert. Diese Makefiles und Patches finden Sie im pkg-Verzeichnis.\cite{RIOTProjectDocumentation}
  
   \vspace{60mm}
  
 \section{Unterst�tzung}
 
  Momentan unterst�tzt der RIOT-Kernel gem�ss Angaben auf der Projektseite \cite{RIOTProjectDocumentation} Prozessoren der Architekturen ARM. Somit  kann es sowohl auf einem System-Emulator wie z.B. Qemu kompiliert werden oder auf einer unterst�tzten Hardware. Folgende Hardware-Hersteller wurden zum Zeitpunkt unserer Projektarbeit unterst�tz:

   
   \begin{center}
   	\begin{tabular}{ | l |}
   		\hline
   		\textbf{Hersteller} 	\\ \hline
   		Arduino		I 			\\ \hline
   		ARM architecture 		\\ \hline
   		Atmel 					\\ \hline
   		Freescale				\\ \hline
   		IoT-LAB					\\ \hline	
   		MSB						\\ \hline	
   		Nordic Semiconductor 	\\ \hline
   		Nucleo					\\ \hline
   		NXP						\\ \hline
   		PIC microcontroller		\\ \hline
   		STMicroelectronics		\\ \hline
   		Texas Instruments		\\ \hline
   		\hline
   	\end{tabular}
   \end{center}
   
 \vspace{4mm}
   
 Zur Kommunikation unterst�tzt das RIOT-Betriebssystem eine Vielzahl an Kommunikations-Protokolle. Da der Fokus auf das Internet der Dinge gelegt wurde, wurden auch die entsprechenden Protokolle zuerst implementiert. 
    
 Folgende Tabelle zeigt die Kommunikationsm�glichkeiten sowohl die vom Betriebssystem unterst�tzte Hardwarekomponenten wie auch die unterst�tzten Protokolle:
    
    
 \begin{center}
  \begin{tabular}{ | l | l |}
		\hline
		\textbf{Protokoll} 		& \textbf{Hardwarekomponente} \\ \hline
		Bluetooth 4.0 			& ADC 			\\ \hline
		CoAP / CSMA / CA		& Ethernet		\\ \hline
		IPv4 / IPv6				& Flash / RAM 	\\ \hline
		IEEE 802.15.4 		 	& GPIO			\\ \hline
		NTP / UNTP				& SPI			\\ \hline
		UDP / UHCP				& UART			\\ \hline
		6LoWPAN 				& 				\\ \hline
		Radios					& 				\\ \hline
		Wi-Fi					& 				\\ \hline
		\hline
   \end{tabular}
  \end{center}




 % !TeX encoding = ISO-8859-1
 \chapter{Contiki-Overview}
 \label{chap:overview}
 
 \section{�bersicht �ber Contiki}
 
  Contiki ist ein Open-Source-Betriebssystem f�r das Internet der Dinge. Contiki verbindet winzige Low-Cost-, Low-Power-Mikrocontroller mit dem Internet. Contiki ist eine leistungsstarke Toolbox f�r den Aufbau komplexer drahtloser Systeme. Die Basis von Contiki und die meisten seiner Kernfunktionen, wie der Kernel, die meisten Bibliotheken, der uIP-Stack und der Rime-Stack wurdne von Adam Dunkels entwickelt.
  
  Contiki bietet einen einfachen ereignisgesteuerten Betriebssystemkern mit sogenannten Protothreads, optionalem Multiprogramming, Interprozesskommunikation via Messagepassing durch Events, eine dynamische Prozessstruktur mit Unterst�tzung f�r das Laden von Programmen, nativen TCP/IP-Support �ber den uIP TCP/IP-Stack und eine grafische Benutzerschnittstelle, welche direkt auf einem Bildschirm oder als virtuelle Anzeige �ber Telnet oder VNC genutzt werden kann. Der Speicherbedraf betr�gt nur wenige Kilobyte und kann f�r extrem eingeschr�nkte Systeme bei Bedarf bis auf einige dutzend Bytes reduziert werden. Inzwischen unterst�tzt Contiki auch IPv6.
  An Anwendungsprogrammen bietet das System einen Webbrowser, einen Web-Server, einen Telnet-Server und vieles mehr.
   
  \begin{figure}[h]
   \centering
   \includegraphics[width=1.0\linewidth]{bilder/Contiki.jpg}
   \caption{�bersicht �ber das Contiki OS}
   \label{fig:components}
  \end{figure}
  
  \vspace{5mm}
  
  Wegen seiner Portabilit�t wurde und wird dieses System an viele Computer angepasst, wie Atari 8-bit-Rechner oder Apple II. Eine der am aktivsten entwickelten Portierungen ist die auf den C64, die sogar eine ebenfalls von Adam Dunkels entwickelte Ethernetanbindung unterst�tzt. PCs k�nnen Contiki ausf�hren und es gibt sogar eine Portierung f�r kleinere Spielekonsolen wie dem Game Boy.
  Daneben findet die Software Einsatz in der Middleware und den Sensoren des RUNES Projekts, Reconfigurable Ubiquitous Networked Embedded Systems.
 
  Der Contiki-Quellcode wird unter einer 3-Klausel BSD-Lizenz ver�ffentlicht. Unter dieser Lizenz kann Contiki frei in kommerziellen und nicht-kommerziellen Systemen verwendet werden, solange das Copyright-Header in den Quellcodedateien beibehalten wird.
  Das Copyright f�r den Contiki-Quellcode liegt im Besitz der Personen oder Organisationen, die es zu Contiki beigetragen haben, aber jeder hat es unter denselben Bedingungen der BSD-Lizenzbestimmungen mitgeteilt.\cite{ContikiProjectDocumentation}
  
 
 \section{Ziele}
 
  Contiki entstand aus dem Wunsch von Adam Dunkels, unerwartete Dinge mit dem Internet zu verbinden. 2001 wurde der Open-Source-UIP-Stack als kleinerer Cousin zu lwIP entwickelt und verbreitete sich rasch �ber die Embedded World. Aber uIP war nicht nur in zahlreichen tief eingebetteten Systemen, sondern auch im Internet der Dinge verwendet. 2003 folgte Contiki.
  Im Jahr 2004 wurde das Konzept der Protothreads, das nun die Grundlage f�r die Prozesse von Contiki bildet, in Contiki eingef�hrt. Fr�he Versionen von Cooja und dem Rime-Stack wurden 2007 mit Contiki 2.0 hinzugef�gt. Power-Profiling wurde 2007 entwickelt. Instant Contiki und das Coffee-Dateisystem wurden Anfang 2008 eingef�hrt. Sp�ter f�hrte Cisco den vollst�ndig zertifizierten IPv6-Stack zu Contiki ein.
  2009 und 2010 wurden viele neue Plattformen zu Contiki hinzugef�gt und neue Low-Power-Mechanismen entwickelt. 2011 wurden zwei wichtige Mechanismen hinzugef�gt: ContikiRPL, f�r IPv6-Routing und ContikiMAC f�r schl�frige Router.
  Im Jahr 2012 wurde Thingsquare gegr�ndet, um Contiki in die Cloud zu bringen.\cite{ContikiProjectDocumentation}
  
  
 \section{Aufbau}
 
 Contiki bietet leistungsstarke Low-Power-Internet-Kommunikation. Contiki unterst�tzt vollst�ndig Standard IPv6 und IPv4, zusammen mit den j�ngsten Low-Power-Wireless-Standards: 6lowpan, RPL, CoAP. Mit Contiki's ContikiMAC und schl�frigen Routern k�nnen sogar drahtlose Router batteriebetrieben werden.\cite{ContikiProjectDocumentation}
 
   \begin{figure}[h]
   	\centering
   	\includegraphics[width=0.7\linewidth]{bilder/Contiki.png}
   	\caption{Struktur des Contiki OS}
   	\label{fig:components}
   \end{figure}

  \vspace{5mm}
 
 Memory Allocation:
 
 Contiki ist f�r kleine Systeme mit nur wenigen Kilobyte Speicherplatz ausgelegt. Contiki ist daher hochspeicherf�hig und stellt einen Satz von Mechanismen f�r die Speicherzuteilung bereit: Speicherblockzuteilung Membran, einen verwalteten Speicherzuweiser mmem sowie den Standard-C-Speicherzuweiser malloc.
 
  \vspace{5mm}
 
 Power Awareness:
 
 Contiki ist f�r den Betrieb in extrem energiesparenden Systemen ausgelegt: Systeme, die m�glicherweise Jahre an einem Paar AA-Batterien betrieben werden m�ssen. Um die Entwicklung von energiesparenden Systemen zu unterst�tzen, bietet Contiki Mechanismen zur Sch�tzung des Energieverbrauchs des Systems und zum Verst�ndnis, wo die Energie verbraucht wurde.
 
  \vspace{5mm}
 
 6lowpan, RPL, CoAP:
 
 Contiki unterst�tzt die k�rzlich standardisierten IETF-Protokolle f�r Low-Power IPv6-Netzwerke, einschlie�lich der 6lowpan-Adaptionsschicht, des RPL-IPv6-Multi-Hop-Routingprotokolls und des CoAP RESTful-Anwendungsschichtprotokolls.
 
 \vspace{5mm}
 
 Dynamic Module Loading:
 
 Contiki unterst�tzt dynamisches Laden und Verkn�pfen von Modulen zur Laufzeit. Dies ist n�tzlich bei Anwendungen, bei denen das Verhalten nach der Implementierung ge�ndert werden soll. Der Contiki-Modul-Loader kann Standard-ELF-Dateien laden, verlagern und verkn�pfen, die wahlweise von den Debugging-Symbolen entfernt werden k�nnen, um deren Gr��e beizubehalten.
 
  \vspace{5mm}
 
 Memory Footprint:
 
 Contiki ist entworfen worden, um in kleinen Mengen des Ged�chtnisses laufen zu lassen. Ein typisches System mit voller IPv6-Vernetzung mit schl�frigen Routern und RPL-Routing ben�tigt weniger als 10 k RAM und 30 k ROM.
 

 \section{Unterst�tzung}

 Momentan unterst�tzt der Contiki-Kernel gem�ss Angaben auf der Projektseite \cite{ContikiProjectDocumentation} Prozessoren der Architekturen ARM. Folgende Hardware-Hersteller wurde zum Zeitpunkt unserer Projektarbeit unterst�tz:
 
\begin{center}
  \begin{tabular}{ | l |}
	\hline
	\textbf{ARM} 				\\ \hline
	Atmel AVR			 		\\ \hline
	Microchip pic32mx795f512l 	\\ \hline
	Microchip pic32mx795f512l	\\ \hline
	nRF52832 					\\ \hline
	RL78						\\ \hline
	TI CC2538					\\ \hline	
	TI CC2538					\\ \hline	
	TI MSP430 					\\ \hline
	\hline
 \end{tabular}
\end{center}
\vspace{4mm}

 Zur Kommunikation unterst�tzt das Contiki-Betreibssystem eine Vielzahl an Kommunikations-Protokolle. Da der Fokus auf das Internet der Dinge gelegt wurde, wurden auch die entsprechenden Protokolle zuerst implementiert. 
 
 Folgende Tabelle zeigt die Kommunikationsm�glichkeiten sowohl die vom Betriebssystem unterst�tzte Hardwarekomponenten wie auch die unterst�tzten Protokolle:
 
 
 \begin{center}
 	\begin{tabular}{ | l | l |}
 		\hline
 		\textbf{Protokoll} 		& \textbf{Hardwarekomponente} \\ \hline
 		Bluetooth 4.0 			& GPIO 		\\ \hline
 		CoAP					& I2C		\\ \hline
 		HTTP					& SPI		\\ \hline
 		IPv4 / IPv6				& UART	 	\\ \hline
 		IEEE 802.15.4 		 	& 			\\ \hline
 		TCP						& 			\\ \hline
 		UDP / UHCP				& 			\\ \hline
 		6LoWPAN 				& 			\\ \hline
 		Radios					& 			\\ \hline
 		RPL						& 			\\ \hline
 		Wi-Fi					& 			\\ \hline
 		\hline
 	\end{tabular}
 \end{center}
 
 \vspace{5mm}
 

 
 
 

 

 

 
 

 

 % !TeX encoding = ISO-8859-1
 \chapter{NuttX Real-Overview}
 \label{chap:overview}
 
 \section{�bersicht �ber NuttX Real}
 
 
 NuttX ist ein Echtzeitbetriebssystem (RTOS) mit einem Schwerpunkt auf Normenkonformit�t und geringem Platzbedarf. Skalierbar von 8-Bit- bis 32-Bit-Mikrocontroller-Umgebungen sind die prim�ren Normen in NuttX die Posix- und ANSI-Standards. Zus�tzliche Standard-APIs von Unix und anderen g�ngigen RTOS, wie VxWorks, werden f�r die Funktionalit�t, die unter diesen Standards nicht verf�gbar ist, oder f�r Funktionalit�ten, die f�r tief eingebettete Umgebungen wie zum Beispiel ''fork'' nicht geeignet sind, �bernommen.
 NuttX wurde zuerst 2007 von Gregory Nutt unter der permissiven BSD Lizenz ver�ffentlicht.
 
 

 NuttX, wie bei vielen RTOS, ist eine Sammlung von verschiedenen Funktionen geb�ndelt als Bibliothek. Es wird nicht ausgef�hrt ausser wenn entweder:
 
  \begin{itemize}
  	\item Die Anwendung den NuttX-Bibliothekscode aufruft
  	\item Ein Interrupt auftritt

  \end{itemize}
  \vspace{5mm}
   
 

 Es gibt einige RTOS-Funktionen, die durch interne Threads implementiert werden.
 
 Eines Scheduler: Diese Logik, die steuert, wenn Tasks oder Threads ausgef�hrt werden. Tats�chlich,
 mehr als das; Der Scheduler wirklich bestimmt, was eine Aufgabe oder ein Thread ist! 
 
 In NuttX ist ein Thread eine beliebige steuerbare Sequenz der Befehlsausf�hrung, die einen eigenen Stack hat. Jeder Aufgabe wird durch eine Datenstruktur dargestellt, die als Task-Steuerblock oder TCB bezeichnet wird. Diese TCBs werden in Listen aufbewahrt. Der Zustand einer Task wird sowohl im Feld ''Task State'' angezeigt. Die meisten dieser Listen sind priorisier. So dass eine gemeinsame Listenhandhabungslogik verwendet werden kann.
 

 Um ein Echtzeit-OS zu sein, muss ein RTOS ''SCHED FIFO'' unterst�tzen. Das hei�t, strenge Priorit�t Scheduling.
''
 Der Thread mit der h�chsten Priorit�t l�uft Zeitraum. Der Thread mit der h�chsten Priorit�t ist immer
 
 Die dem TCB am Anfang der readytorun-Liste zugeordnet ist. NuttX unterst�tzt eine weitere Echtzeit-Planungsrichtlinie: ''SCHED RR''. Die RR steht f�r Roundrobin
 Und dies wird manchmal als Round-Robin-Scheduling bezeichnet. In diesem Fall unterst�tzt NuttX das Timelicing. Eine Aufgabe mit ''SCHED RR'' Scheduling-Richtlinie ausgef�hrt wird, dann, wenn jedes timeslice vergeht, wird es aufgeben
 Die CPU auf die n�chste Task, die die gleiche Priorit�t hat. 

 Jede Aufgabe wird nicht nur durch eine TCB, sondern auch durch eine numerische Task-ID dargestellt. Angesichts einer Task-ID, das RTOS
 Kann das TCB finden; Bei einem TCB kann das RTOS die Task-ID finden. Sie sind also funktional gleichwertig. Nur die Task-ID wird jedoch an den RTOS / Application Interfaces freigelegt.
 
 
 1.3 NuttX Aufgaben
 1.3.1Prozesse und Threads
 In gr��eren System-OS wie Windows oder Linux, werden Sie oft hier die Namen Prozesse verwendet, um zu verweisen
 Threads, die vom Betriebssystem verwaltet werden. Ein Prozess ist mehr als ein Thread, wie wir bisher diskutiert haben. EIN
 Prozess ist eine gesch�tzte Umgebung, die einen oder mehrere Threads hostet. Unter Umgebung verstehen wir den Satz von
 Ressourcen von der OS beiseite, aber im Falle der gesch�tzten Umgebung des Prozesses sind wir
 Spezifisch mit seinem Adressraum.
 Um den Adressraum des Prozesses zu implementieren, muss die CPU eine Speicherverwaltungseinheit unterst�tzen
 (MMU). Die MMU wird verwendet, um die gesch�tzte Prozessumgebung zu erzwingen.
 Allerdings wurde NuttX entworfen, um die mehr Ressourcen beschr�nkt, niedrigere Ende, tief eingebettet unterst�tzen
 MCUs. Diese MCUs haben selten eine MMU und k�nnen daher Prozesse niemals unterst�tzen
 Unterst�tzung von Windows und Linux. So unterst�tzt NuttX keine Prozesse. NuttX unterst�tzt eine MMU
 Aber es wird nicht die MMU verwenden, um Prozesse zu unterst�tzen. NuttX arbeitet nur in einem flachen Adre�raum. (NuttX
 Wird die MMU verwenden, um die Befehls- und Datencaches zu steuern und gesch�tzte Speicherbereiche zu unterst�tzen).
 1.3.2 NuttX Aufgaben und Aufgabenressourcen
 Alle RTOS unterst�tzen den Begriff einer Aufgabe. Eine Aufgabe ist das moralische �quivalent eines Prozesses des RTOS. Wie ein
 Prozess ist eine Aufgabe ein Thread mit einer damit verkn�pften Umgebung. Diese Umgebung ist wie
 Umfeld des Prozesses, enth�lt aber keinen privaten Adressraum. Diese Umgebung ist privat
 Und einzigartig f�r eine Aufgabe. Jede Aufgabe hat ihre eigene Umgebung
 Diese Aufgabenumgebung besteht aus einer Anzahl von Ressourcen (wie im TCB dargestellt). Interessiert an
 Diese Diskussion sind die folgenden. Beachten Sie, dass eine dieser Taskressourcen im NuttX deaktiviert werden kann
 Konfiguration, um die NuttX-Speicherfl�che zu reduzieren:
 
 
 
 

 
 
 
 
 

 
 \vspace{5mm}
 \section{Ziele}
 \vspace{5mm}
  Umgebungsvariablen. Dies ist die Sammlung von variablen Zuordnungen des Formulars:
  VARIABLE = VALUE
  2. Dateideskriptoren. Ein Dateideskriptor ist eine aufgabenbezogene Zahl, die eine offene Ressource darstellt (a
  Datei oder einen Ger�tetreiber).
  3. Steckdosen. Ein Socket-Deskriptor ist wie ein Dateideskriptor, aber die offene Ressource ist in diesem Fall ein
  Netzwerk-Buchse.
  4. Str�me. Str�me stellen Standard-C-gepufferte E / A dar. Streams wickeln Dateideskriptoren oder Sockets ein
  Bieten einen neuen Satz von Schnittstellenfunktionen f�r den Umgang mit der Standard-C-E / A (wie fprintf (),
  Fwrite () usw.).
  
  
  
  
  
  NuttX implementieren ein virtuelles Dateisystem (VFS), das verwendet werden kann, um mit einer Anzahl von zu kommunizieren
  Verschiedene Entities �ber die standardm��igen open (), close (), read (), write (), etc. Schnittstellen. Wie andere
  VFSs unterst�tzt das NuttX VFS Dateisystem-Mountpunkte, Dateien, Verzeichnisse, Ger�tetreiber usw.
  Auch wie bei anderen VFSs unterst�tzt das NuttX-Dateisystem psuedo-Dateisysteme, also Dateisysteme
  Die als normale Medien erscheinen, aber wirklich unter programmatischer Kontrolle pr�sentiert werden. Unter Linux, f�r
  Beispielsweise haben Sie die / proc und die / sys psuedo-Dateisysteme. Es gibt keine physischen Medien
  Das dem Pseudo-Dateisystem zugrunde liegt.
  Das NuttX-Root-Dateisystem ist immer ein Pseudo-Dateisystem. Das ist genau das Gegenteil von Linux. Mit
  Linux muss das Root-Dateisystem immer ein physikalisches Block-Device sein (wenn auch nur ein initrd-RAM-Datentr�ger). Dann
  Sobald Sie das physische Root-Dateisystem installiert haben, k�nnen Sie andere Dateisysteme - einschlie�lich
  Linux psuedo-Dateisysteme wie / proc oder / sys. Bei NuttX ist das Root-Dateisystem immer ein Pseudofile
  System, das keinen zugrunde liegenden Blocktreiber oder physikalisches Ger�t ben�tigt. Dann k�nnen Sie montieren
  Realen Dateisystem im Pseudo-Dateisystem.
  Diese Anordnung macht das Leben viel einfacher f�r die kleine eingebettete Welt (aber auch
  Hat ein paar Einschr�nkungen - wie, wo Sie k�nnen Dateisysteme mount).
  NuttX interagiert mit Ger�ten �ber Ger�tetreiber - das hei�t �ber Software, die Hardware steuert und
  
  

  Systeme wie Linux unterst�tzen auch POSIX pthreads. In der Linux-Umgebung wird der Prozess erstellt
  Mit einem Faden l�uft in ihm. Aber durch die Verwendung von Schnittstellen wie pthread create (), k�nnen Sie erstellen
  Mehrere Threads, die die gleichen Prozessressourcen ausf�hren und gemeinsam nutzen.
  NuttX unterst�tzt auch POSIX pthreads und die NuttX pthreads unterst�tzen auch dieses Verhalten. Das hei�t, die
  NuttX POSIX pthreads teilen sich auch die Ressourcen der �bergeordneten Aufgabe. Allerdings, da NuttX nicht
  Support-Prozess-Adresse Umgebungen, ist der Unterschied nicht so auff�llig. Wenn eine Task einen pthread erzeugt,
  Die neu erstellen pthread teilen die Umgebungsvariablen, Dateideskriptoren, Sockets und Streams von
  Die �bergeordnete Aufgabe.
  Hinweis: Diese Task-Ressourcen werden als Referenz gez�hlt und bleiben so lange bestehen, wie Thread in der Task-Gruppe ist
  immer noch aktiv.
 \vspace{5mm}
 
 \section{Aubau}
 \vspace{5mm}
 
  2. Die NuttX-Initialisierungssequenz
  2.1 �bersicht
  Auf der h�chsten Ebene kann die NuttX-Initialisierungssequenz in drei Phasen dargestellt werden:
  1. Die hardware-spezifische Einschalt-Reset-Initialisierung,
  2. NuttX RTOS Initialisierung, und
  3. Anwendungsinitialisierung.
  Diese Initialisierungssequenz ist wirklich ganz einfach, weil das System im Single-Thread-Modus l�uft
  Bis der Punkt beginnt, der die Anwendung startet. Das bedeutet, dass die Initialisierungssequenz gerade ist
  Einfache, geradlinige Funktionsaufrufe.
  Kurz vor dem Starten der Anwendung geht das System in den Multi-Thread-Modus, und die Dinge k�nnen mehr bekommen
  Komplex.
  Diese werden in den folgenden Abschnitten n�her erl�utert.
  2.2 Einschalten Zur�cksetzen Initialisierung.
  2.2.1 �bersicht
  Die Software beginnt mit der Ausf�hrung, wenn der Prozessor zur�ckgesetzt wird. In der Regel beim Einschalten, aber alle R�ckstellungen sind
  Grunds�tzlich die gleichen, wo sie auftreten, weil der Power-on, dr�cken Sie die Reset-Taste, oder auf einem Watchdog
  Timer-Ablauf. Die Software, die ausgef�hrt wird, wenn der Prozessor zur�ckgesetzt wird, ist f�r die jeweilige CPU eindeutig
  Architektur und ist kein gemeinsamer Bestandteil von NuttX. Die Art der Dinge, die von der
  Architektur-spezifische Reset-Handling beinhaltet:
  1. Setzen des Prozessors in seinen Betriebszustand. Dies kann Dinge wie Einstellung CPU-Modi;
  Initialisierung von Co-Prozessoren usw.
  2. Einstellen der Taktung, so dass die Software und Peripherie wie erwartet funktionieren,
  3. Einrichten des C-Stack-Zeigers (und anderer Prozessorregister)
  4. Speicher initialisieren und
  5. Starten von NuttX.
  2.2.2 Speicherinitialisierung
  In C-Implementierungen gibt es zwei allgemeine Klassen von variablen Speicher. Zuerst gibt es die initialisiert
  Variablen. Betrachten wir z. B. die globale Variable x:
 \vspace{5mm}
 \section{Unterst�tztung}
 
 
 Hauptmerkmale
 ? Normenkonform.
 ? Kernaufgabenmanagement.
 ? Modularer Aufbau.
 ? Vollst�ndig preemptible.
 ? Nat�rlich skalierbar.
 ? Hochkonfigurierbar.
 ? Leicht erweiterbar auf neue Prozessorarchitekturen, SoC-Architekturen oder Platinenarchitekturen. Siehe Portieranleitung.
 ? FIFO, Round-Robin und "sporadische" Terminierung.
 ? Echtzeit, deterministisch, mit Unterst�tzung f�r priorit�re Vererbung.
 ? Tickless Betrieb.
 ? POSIX / ANSI-�hnliche Aufgabensteuerelemente, benannte Nachrichtenwarteschlangen, z�hlende Semaphoren, Taktgeber / Timer, Signale, pthreads, Umgebungsvariablen, Dateisystem.
 ? VxWorks-�hnliche Task-Management und Watchdog-Timer.
 ? BSD-Socket-Schnittstelle.
 ? Erweiterungen zur Verwaltung der Pre-Emption.
 ? Optionale Aufgaben mit Adressumgebungen (Prozesse).
 ? Symmetrisches Mehrprozessorsystem (SMP)
 ? Ladbare Kernelmodule.
 ? Speicherkonfigurationen: (1) Flat Embedded Build, (2) Gesch�tzter Build mit MPU und (3) Kernel-Build mit MMU.
 ? Speicherzuweisungen: (1) Standardhapspeicherbelegung, (2) granul�rer Zuweiser, (3) gemeinsam genutzter Speicher und (4) dynamisch bemessene Proze�haufen.
 ? Thread Lokaler Speicher (TLS)
 ? Vererbbare "Steuerklemmen" und E / A-Umleitung. Pseudo-Anschl�sse.
 ? On-Demand-Paging.
 ? Systemprotokollierung
 ? Kann entweder als offenes, flaches eingebettetes RTOS oder als eigenst�ndiger, sicherer Kernel mit einer System-Call-Gate-Schnittstelle aufgebaut werden.
 ? Integrierte, pro-thread CPU-Lastmessungen.
 ? Benutzerdefinierte NuttX C-Bibliothek
 ? Gut dokumentiert im NuttX Benutzerhandbuch.
 % !TeX encoding = ISO-8859-1
 \chapter{Vergleich}
 \label{chap:overview}
 
 \section{�bersichttabelle}
 
 \begin{center}
 	\begin{tabular}{ | l | l | l | l | l | l |}
 		\hline
 		\textbf{ARC}		& \textbf{ARM} 		& \textbf{X86}  \\ \hline
 		Arduino 101			& 96B-Carbon		& Galileo Gen1 Gen2   \\ \hline
 		EDesignWare EM		& 96B-Nitrogen		& Minnowboard Max \\ \hline
 		Emulation / QEMU	& Arduino Due		& Quark D2000 CRB	\\ \hline
 		& Hexiwear		 	& Emulation / QEMU	\\ \hline
 		& NXP FRDM-K64		&					\\ \hline
 		& OLIMEXINO-STM32	&					\\ \hline	
 		& nRF51-PCA10028	&					\\ \hline
 		& nRF52-PCA10040	&					\\ \hline
 		& nRF52840-PCA10056	&					\\ \hline
 		& V2M Beetle		&					\\ \hline
 		& Emulation / QEMU	&					\\ 
 		\hline
 	\end{tabular}
 \end{center}
% !TeX encoding = ISO-8859-1
\chapter{nRF52DK}
\label{chap:nrf52dk}

\section{Aufsetzen der nordic SDK unter Ubuntu}
\section{Vorteile}
\section{Technische Beschreibung des Boards}
\section{Bluetooth Low Energy}
\section{Analog to Digital Converter}
\section{I2C Bus}
% !TeX encoding = ISO-8859-1
\chapter{DemoApp}
\label{chap:demoapp}

\section{Technische �bersicht}
\section{Hardware Dokumentation}
\section{Software Dokumentation}
\section{Softwaretests}
% !TeX encoding = ISO-8859-1
\chapter{Fazit}
\label{chap:fazit}

\section{Vergleich Ist/Soll}
\section{W�nschenswerte Erweiterungen}
%---------------------------------------------------------------------------

% Selbst�ndigkeitserkl�rung
%---------------------------------------------------------------------------
\cleardoublepage
\phantomsection 
\addcontentsline{toc}{chapter}{Selbst�ndigkeitserkl�rung}
\include{vorspann/selbstaendigkeitserklaerung}
%---------------------------------------------------------------------------

% Glossary
%---------------------------------------------------------------------------
\phantomsection 
\addcontentsline{toc}{chapter}{Glossar}
% !TeX encoding = ISO-8859-1
\chapter*{Glossar}
\label{chap:Glossar}
% \ac{} normal erste verwendung in langform, ansonsten in kurzform
% \acs{} erzwingt kurzform
% \acl{} erzwingt langform
% \acp{} plural form
\begin{acronym}[HTTPS] %[ ] l�ngste Abk�rzung f�r einr�cken
	\acro{API}{Application Programming Interface}
	\acro{CPU}{Central Processing Unit}
	\acro{RTOS}{Realtime Operating Systems}
	\acro{HTTPS}{Hypertext Transfer Protocol Secure}
	\acro{BLE}{Bluetooth Low-Energy}
	\acro{NFC}{Near Field Communication}
	\acro{IoT}{Internet der Dinge}
\end{acronym}
%\printglossary
%---------------------------------------------------------------------------

% Bibliography
%---------------------------------------------------------------------------
\cleardoublepage
\phantomsection
\addcontentsline{toc}{chapter}{Literaturverzeichnis}
\bibliographystyle{IEEEtranS}
\bibliography{datenbanken/bibliography}{}
%---------------------------------------------------------------------------

% Listings
%---------------------------------------------------------------------------
\cleardoublepage
\phantomsection
\addcontentsline{toc}{chapter}{Abbildungsverzeichnis}
\listoffigures
\cleardoublepage
\phantomsection
\addcontentsline{toc}{chapter}{Tabellenverzeichnis}
\listoftables
%---------------------------------------------------------------------------

% Index
%---------------------------------------------------------------------------
\cleardoublepage
\phantomsection
\addcontentsline{toc}{chapter}{Stichwortverzeichnis}
\renewcommand{\indexname}{Stichwortverzeichnis}
\printindex
%---------------------------------------------------------------------------

% Attachment:
%---------------------------------------------------------------------------
\appendix
\settocdepth{section}
% !TeX encoding = ISO-8859-1
\chapter{Beliebiger Anhang}
\label{chap:bel_anhang}
\include{anhang/beispielanhangB}
%---------------------------------------------------------------------------

\end{document}