% !TeX encoding = ISO-8859-1
\chapter*{Management Summary}
\label{chap:managementSummary}

% Abstract (eingeben):
%---------------------------------------------------------------------------

Diese Arbeit bietet einen �berblick �ber das Betriebssystem Zephyr mit Schwerpunkt auf Technologien, Protokolle und Anwendungsfragen. Das Zehphyr Beriebssystem ist speziell f�r das Internet der Dinge (IoT), welches durch die neuesten Entwicklungen in den Bereichen RFID, intelligente Sensoren, Kommunikationstechnologien und Internet-Protokolle schnell w�chst, entwickelt worden. Die Grundvoraussetzung ist, dass intelligente Sensoren direkt ohne menschliches Engagement zusammenarbeiten, um eine neue Klasse von Anwendungen zu liefern. Die aktuelle Revolution in den Bereichen Internet, Mobile und Machine-to-Machine (M2M) ist die erste Phase des IoT. In den kommenden Jahren wird erwartet, dass die IoT diverse Technologien �berbr�ckt, um neue Anwendungen durch die Verbindung von physischen Objekten zur Unterst�tzung intelligenter Entscheidungsfindung zu erm�glichen.  Die Linux Foundation hat mit dem Projekt Zephyr mit der Entwicklung eines Echtzeit-Betriebssystems f�r das Internet der Dinge (IoT) begonnen. Zephyr ist ein Open-Source-Betriebssystem mit dem Ziel ein solides OS f�r IoT Ger�te mit geringen Ressourcen bereitzustellen. Es nutzt eine echtzeitf�hige Kombination aus Nano- und Microkernel. Im Gegensatz zu einem Linux Kernel ben�tigt Zephyr nur zwischen 8 und 512 KByte an Arbeitsspeicher.  Aktuell werden folgende Plattformen unterst�tzt: x86, ARM und ARC.

Diese Arbeit beginnt mit einem horizontalen �berblick �ber das Zehphyr Beriebssystem. Anschliessend wird detailiert der Aufbau und die Ziele des Betriebsysstems erl�utert und einen �berblick �ber einige technische Details, wie z.B. Technologien, Protokolle und Anwendungen gew�hrt. Im Vergleich zu anderen Berichten, sofern diese schon vorliegen in diesem Bereich, ist es unser Ziel, eine umfassendere Zusammenfassung der wichtigsten Protokolle und Anwendungsprobleme zu bieten, damit Forscher und Anwendungsentwickler einen schnellen �berblick �ber die gew�nschten Funktionalit�ten erhalten. Des Weiteren bieten wir auch einen �berblick �ber einige andere RTOS an, um die Vor- und Nachteile des Zephyr Betriebssystems besser auf zu zeigen. Ausserdem schrieben wir eine Anleitung. Wie genau man unser verwendetes Noridc Eval-Board in Betrieb nehmen kann, wie unsere Applikation aufgebauen ist und wie man sie bedient. Das Ziel der Arbeit
war, mit dem neuen Betreibsystem Zephyr Erfahrungen zu sammeln. Zu evaluieren, welches Betriebssystem sich am besten f�r das Internet der Dinge (IoT) und die passenden Sensorknoten eignet, um diese Erkenntnisse in sp�teren Projekten der Berner Fachhochschule einsetzen zu k�nnen.

 

