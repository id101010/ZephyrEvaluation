% !TeX encoding = ISO-8859-1
\chapter*{Management Summary}
\label{chap:managementSummary}

% Abstract (eingeben):
%---------------------------------------------------------------------------

Diese Arbeit bietet einen �berblick �ber das Betriebssystem Zephyr mit Schwerpunkt auf Technologien, Protokollen und Anwendungsfragen. Das Zephyr Betriebssystem wird mit Fokus auf das Internet der Dinge (IoT) entwickelt. Durch die neuesten Entwicklungen in den Bereichen RFID, intelligente Sensoren, Kommunikationstechnologien und Internet-Protokolle w�chst dieses rasant. Die Grundvoraussetzung ist, dass intelligente Sensoren direkt ohne menschliches Engagement zusammenarbeiten, um eine neue Klasse von Anwendungen zu liefern. Die aktuelle Revolution in den Bereichen Internet, Mobile und Machine-to-Machine (M2M) ist die erste Phase des IoT. In den kommenden Jahren wird erwartet, dass das IoT diverse Technologien �berbr�ckt, um neue Anwendungen durch die Verbindung von physischen Objekten zur Unterst�tzung intelligenter Entscheidungsfindung zu erm�glichen.  Die Linux Foundation unterst�tzt die Entwicklung des Projektes Zephyr formal als Echtzeit-Betriebssystem f�r das Internet der Dinge. Zephyr ist vollst�ndig Open-Source und verfolgt das Ziel ein solides OS f�r IoT Ger�te mit geringen Ressourcen bereitzustellen. Es nutzt eine echtzeitf�hige Kombination aus Nano- und Microkernel. Im Gegensatz zu einem Linux Kernel ben�tigt Zephyr nur zwischen 8 und 512 KByte an Arbeitsspeicher.  Aktuell werden folgende Architekturen unterst�tzt: x86, ARM und ARC.

Diese Arbeit beginnt mit einem horizontalen �berblick �ber das Zephyr Betriebssystem. Anschliessend wird detailliert der Aufbau und die Ziele des Betriebssystems erl�utert und einen �berblick �ber einige technische Details, wie z.B. Technologien, Protokolle und Anwendungen gew�hrt. Im Vergleich zu anderen Berichten ist unser Ziel, eine umfassende Zusammenfassung der wichtigsten Protokolle und Anwendungsprobleme zu bieten, damit Forscher und Anwendungsentwickler einen schnellen �berblick �ber die gew�nschten Funktionalit�ten erhalten. Weiter bieten wir auch einen �berblick �ber einige andere RTOS an, um die Vor- und Nachteile des Zephyr Betriebssystems besser aufzuzeigen. 
Im Bericht enthalten ist auch eine Anleitung, wie man das von uns evaluierte Noridc Eval-Board nRF52DK unter Zephyr in Betrieb nimmt und wie Zephyr-Applikation aufgebaut sind. 
Das Ziel der Arbeit war, mit Zephyr Erfahrungen zu sammeln, eine Hardware f�r Zephyr zu evaluierend und eine Applikation in Betrieb zu nehmen. Die gewonnen Erkenntnisse k�nnen in sp�teren Projekten der Berner Fachhochschule eingesetzt werden.

 

